
%\documentclass[conference]{IEEEtran}
\documentclass[10pt,conference]{IEEEtran}

\pagestyle{plain}

\usepackage[english,american]{babel}
\usepackage{graphicx}
\usepackage{subfigure}
\usepackage{amsmath}
\usepackage{multirow}
\usepackage{multicol}
\usepackage{float}
\usepackage{algorithm}
\usepackage{algorithmic}
\usepackage[colorlinks, linkcolor=red, anchorcolor=green, citecolor=blue]{hyperref}
\usepackage{cite}
\usepackage{balance}
\usepackage{color}
\usepackage[square,sort,comma,numbers]{natbib}
\usepackage{url}
\usepackage{diagbox}
\usepackage{enumerate}
\usepackage{setspace}
\let\labelindent\relax
\usepackage{enumitem}
\usepackage{indentfirst}
\usepackage{booktabs}
\usepackage{tikz}
\usepackage{listings}
\usepackage{etoolbox}
\usepackage{setspace}
\usepackage{amsfonts}

\hyphenation{op-tical net-works semi-conduc-tor}
%\renewcommand{\baselinestretch}{0.985}
%\renewcommand{\captionfont}{\linespread{1.5}\normalsize}

%\renewcommand{\thesection}{\arabic{section}}
%\renewcommand{\thesubsection}{\thesection.\arabic{subsection}}
%\renewcommand{\thesubsubsection}{\thesubsection.\arabic{subsubsection}}

%\makeatletter
%\def\@seccntformat#1{\@ifundefined{#1@cntformat}%
%   {\csname the#1\endcsname\quad}%       default
%   {\csname #1@cntformat\endcsname}}%    enable individual control
%\newcommand\section@cntformat{}
%\makeatother

\renewcommand{\algorithmicrequire}{\textbf{Input:}}
\renewcommand{\algorithmicensure}{\textbf{Output:}}
%\usepackage{setspace}
%\usepackage{epsfig,graphics,subfigure,psfrag,amsmath,amssymb}
\newcommand\FIXME[1]{\textcolor{red}{FIX:}\textcolor{red}{#1}}
\newcommand\FIXED[1]{\textcolor{blue}{FIXED: }\textcolor{blue}{#1}}

\newcommand{\tabincell}[2]{\begin{tabular}{@{}#1@{}}#2\end{tabular}}

%\def\@IEEEsectpunct{.\ \,}
%\def\paragraph{\@startsection{paragraph}{4}{\z@}{1.5ex plus 1.5ex minus 0.5ex}%
%{0ex}{\normalfont\normalsize\sffamily\bfseries}}


\newcommand{\circled}[2][]{\tikz[baseline=(char.base)]
    {\node[shape = circle, draw, inner sep = 1pt]
    (char) {\phantom{\ifblank{#1}{#2}{#1}}};%
    \node at (char.center) {\makebox[0pt][c]{#2}};}}
\robustify{\circled}

\newcommand{\circling}[2][]{\tikz[baseline=(char.base)]
    {\node[shape = circle, draw = gray!100, inner sep = 1pt, fill= gray!100]
    (char) {\phantom{\ifblank{#1}{#2}{#1}}};%
    \node at (char.center) {\makebox[0pt][c]{#2}};}}
\robustify{\circled}



\begin{document}
%\setcopyright{acmcopyright}

\title{When Deep Learning Meets Captcha: A Large Scale Study Showing That Captcha Should Be Abandoned}
\author{
}

\IEEEoverridecommandlockouts \makeatletter\def\@IEEEpubidpullup{9\baselineskip}\makeatother \IEEEpubid{\parbox{\columnwidth}{ }
\hspace{\columnsep}\makebox[\columnwidth]{}}

\maketitle

\begin{abstract}
CAPTCHA is a standard secure mechanism to automated tell computers and humans, and is widely applied to prevent against malicious bot programs. In this study, we carry out a large scale systematic study on current text-based Captchas based on complex deformable characters that are augmented with both anti-segmentation and anti-recognition techniques. Unlike prior fine research that focus on the text-based Captchas with the same or similar styles, we present a novel generic hierarchical attack on different styles of Captchas. Our approach is able to translate these various styles of Captchas to a unique one using conditional Generative Adversarial Networks (cGANs). Using our CNN recognition engine, we can accurately identify the translated Captchas. We thoroughly evaluated our approach using real-world Captchas, and found all current text-based Captchas are vulnerable to our automated attack with an average speed of 224.35ms on an ordinary server (with a 3.2-GHz Inter Xeon CPU with 100-GB RAM and a TITAN Xp GPU). We also discovered that, the Captcha with complex noisy background do not defend against segmentation under our attack. This is demonstrated by the fact that we are able to clean out any sophisticated background from the Captcha image. Since our attack can break a widely of current text-based Captchas, this paper calls the community to revisit the risks of using text-based Captcha to defend against malicious bots.
\end{abstract}

\begin{IEEEkeywords}
    Captcha, Security, Generic Attack, Text-based
\end{IEEEkeywords}


%TODO:
% Read: A pilot study on the security of pattern screen-lock methods and soft side channel attacks
\begin{figure*}[!t]
            \centering
            \subfigure{
                \begin{minipage}[t]{0.2\textwidth}
                    \includegraphics[width=\textwidth]{fig/prior_captchas/prior_captcha11.png}\\
                    \center (a) Character isolated Captcha
                \end{minipage}
            }
            \hspace{0.16cm}
            \subfigure{
                \begin{minipage}[t]{0.20\textwidth}
                    \includegraphics[width=\textwidth]{fig/prior_captchas/prior_captcha2.png}\\
                    \center (b) Megaupload
                \end{minipage}
            }
            \hspace{0.16cm}
            \subfigure{
                \begin{minipage}[t]{0.20\textwidth}
                    \includegraphics[width=\textwidth]{fig/prior_captchas/prior_captcha3.png}\\
                    \center (c) Yahoo!
                \end{minipage}
            }
            \hspace{0.16cm}
            \subfigure{
                \begin{minipage}[t]{0.20\textwidth}
                    \includegraphics[width=\textwidth]{fig/prior_captchas/prior_captcha4.png}\\
                    \center (d) Recaptchas
                \end{minipage}
            }
            \hspace{0.16cm}
            \subfigure{
                \begin{minipage}[t]{0.20\textwidth}
                    \includegraphics[width=\textwidth]{fig/prior_captchas/prior_captcha7.png} \\
                    \center (e) Skyrock
                \end{minipage}
            }
            \hspace{0.16cm}
            \subfigure{
                \begin{minipage}[t]{0.20\textwidth}
                    \includegraphics[width=\textwidth]{fig/prior_captchas/prior_captcha8.png} \\
                    \center (f) Digg
                \end{minipage}
            }
            \hspace{0.16cm}
            \subfigure{
                \begin{minipage}[t]{0.20\textwidth}
                    \includegraphics[width=\textwidth]{fig/prior_captchas/prior_captcha9.png}\\
                    \center (g) Reddit
                \end{minipage}
            }
            \hspace{0.16cm}
            \subfigure{
                \begin{minipage}[t]{0.20\textwidth}
                    \includegraphics[width=\textwidth]{fig/prior_captchas/prior_captcha10.png}\\
                    \center (h) Captcha.net
                \end{minipage}
            }
            \subfigure{
                \begin{minipage}[t]{\textwidth}
                    \center (1) Captchas used in previous attacks.\\
                \end{minipage}
            }
            \hspace{0.16cm}
            \subfigure{
                \begin{minipage}[t]{0.20\textwidth}
                    \includegraphics[width=\textwidth]{fig/prior_captchas/prior_captcha11.jpg} \\
                    \center (i) Baidu
                \end{minipage}
            }
            \hspace{0.16cm}
            \subfigure{
                \begin{minipage}[t]{0.20\textwidth}
                    \includegraphics[width=\textwidth]{fig/prior_captchas/prior_captcha12.jpg} \\
                    \center (j) Tecent
                \end{minipage}
            }
            \hspace{0.16cm}
            \subfigure{
                \begin{minipage}[t]{0.20\textwidth}
                    \includegraphics[width=\textwidth]{fig/prior_captchas/prior_captcha13.jpg}\\
                    \center (k) NetEase
                \end{minipage}
            }
            \hspace{0.16cm}
            \subfigure{
                \begin{minipage}[t]{0.20\textwidth}
                    \includegraphics[width=\textwidth]{fig/prior_captchas/prior_captcha15.png}\\
                    \center (l) Google
                \end{minipage}
            }
            \subfigure{
                \begin{minipage}[t]{\textwidth}
                    \center (2) Current Captchas deployed in some popular websites.\\
                \end{minipage}
            }
            \caption{This shows the text-based Captcha schemes used in prior works and current Captchas deployed in some famous websites. The scheme in (a) shows the character isolated Captcha that is deployed originally.}
            \label{fig:text-based captchas}
        \end{figure*}
\section{Introduction}

A CAPTCHA (Completely Automated Public Turing Test to Tell Computers and Human Apart) is a automated test that humans can pass but computer programs cannot~\cite{Von2004Telling}. It provides a effective approach for automatically distinguishing humans from computer systems, and therefore is used to defend against automatic spam, registration or malicious bots~\cite{Von2003CAPTCHA,Tam2008Breaking}.

The most widely deployed CAPTCHA is the so-called text-based scheme~\cite{Yan2008Usability}, which mainly consists of distorted English letters and Arabic numerals. The popularity of this scheme is due to its obvious advantages~\cite{Chellapilla2005Building,Chellapilla2005Computers}: (1) most people around the world can recognize English letters and Arabic numerals; (2) the space of the text-based Captcha are huge so that the brute-force attack can be defeated. Given its pervasive usage, a security breach of the text-based Captcha could lead to serious consequences.

The robustness of text-based Captchas is the significant concern in the research communities. Over the past decade, researchers have uncovered a number of ways to recognize text-based Captchas. Many fine prior researches just focus on attacking an unique Captcha scheme~\cite{Gao2013The,Gao2017Research,Mohamed2014A,Yan2008A}. This limits their applicability. Recently the generic attacks have been proposed by Gao \emph{et al.}~\cite{Gao2016A} and Bursztein \emph{et al.} ~\cite{Bursztein2011Text,Bursztein2014The}. They claim that they can break a wide range of text-based Captchas using a generic attacking method. However, these defeated Captchas possess relative simple noisy background or uniform font style. Although the security of text-based Captchas have been proven frail, many companies such as Google, Microsoft and Baidu still use such scheme as current text-based Captcha scheme have more complex background or distorted characters. This makes previous attacks invalid.  Recent studies~\cite{Thomas2013Trafficking,Bursztein2014Easy} also demonstrate that text-based Captcha is still a secure mechanism.

The key factor of previous attacks on text-based Captcha lies in the difficulty of segmenting characters~\cite{Chellapilla2005Computers}. Therefore, the basic design principle of text-based Captchas should be anti-segmentation. To this end, current text-based Captchas with more complex noisy background and distorted characters have been proposed and deployed by many companies. In order to increase the difficulty of finding where each character is, such Captcha shorten the distance between characters. All attacks proposed above were invalid due to cannot successfully segment the characters. It is therefore driving us wondering: is current sophisticated Captcha scheme as secure as it is expected? This fundamental question precipitated our study.

In this paper, we present a novel generic attack on current text-based Captchas using deep learning technology. Our attack employs a variant of the conditional generative adversarial networks (cGANs)~\cite{pix2pix2016} to transform the distorted Captchas to the regular ones.
To do so, we developed a hierarchical generative model using the cGANs. The hierarchical generative mode consists of three components: background removing model, segmentation model and regularization model.
The background removing model is used to clean out the complicated noisy background of Captchas and outputs the clean Captchas. The segmentation model aims to expand the space between the adjacent characters of the clean Captchas. This step outputs the Captchas with larger inter-character space. Note that until this step, the characters of Captchas are still distorted. Such distorted characters can be translated to the regular ones by our regularization model.
At last, the transformed regular Captchas are recognized by our recognition engineer trained using the Convolutional Neural Network (CNN).

We thoroughly evaluate our approach using real-world captchas collected from some websites. We show that our approach is effective in recognizing current text-based Captchas and as a results, we can defeat almost all current text-based captchas with a success rate range from XX\% to XX\%. We demonstrate that, the sophisticated noisy background cannot offer stronger protection in term of anti-segmentation under our attack. Our finding suggests that text-based captchas are insecure under the age of artificial intelligence.


\textbf{Contributions} This paper makes the following specific contributions:

\begin{itemize}
  \item[-] \textbf{A Large Scale Evaluation.} We conduct a large scale evaluation on real-world text-based Captchas. Our experimental results show that all current Captchas deployed in famous websites are vulnerable to automated attacks.
  \item[-] \textbf{A New Segmentation technique.} We design a hierarchical generation models that can effectively segment the deformed characters on the Captchas.
  \item[-] \textbf{New Findings.} Our study suggests that complex noisy background cannot strength the security of Captchas.
\end{itemize} 
\begin{figure*}[!t]
  \centering
  \includegraphics[width=0.9\textwidth]{fig/overview.pdf}
  \caption{The overview of our attack.}
  \label{fig:overview}
\end{figure*}

\section{Background}

\subsection{Text-based Captchas}

CAPTCHA is also called inverse Turing test~\cite{Naor1996Verification}, and aims to determine whether or not the user is human. The most popularity of Captchas is text-based Captchas. Initially, they were composed of deformed English letters and Arabic numerals which human can recognize while computers cannot. The usage of English letters and Arabic numerals makes text-based Captchas being welcomed world-wide~\cite{Chellapilla2005Building,Chellapilla2005Computers}. Due to such advantages, text-bsaed Captchas have been widely used in various of Internet security access tasks~\cite{Von2004Telling}. These tasks include defending against malicious scraping web contents, preventing automatic registration of free accounts for spam posting or forums, or mitigating the impact of DDoS attacks.

\subsubsection{Captcha Security Features} \label{section: sccturity_features}
To summary out a representative security features, we collected a wide range of Captchas that had ever been or being used by the top 20 most popular web sites which are listed by Alexa~\footnote{http://www.alexa.com/topsites}. Figure~\ref{fig:text-based captchas} gives some representative examples of our collected Captchas~\footnote{Some previous used Captchas come from the related works~\cite{Gao2013The,Gao2016A,Bursztein2011Text}.}. It obviously shows that current text-based Captchas have more complex security features than previous Captchas. These features can be classified into two categories: anti-segmentation and anti-recognition features.
The anti-segmentation feature aims to increase the difficulty of character segmentation. In order to implement such feature, the Captcha is needed to be confused on the whole with sophisticated background, extra connecting lines and overlapping characters. Thus, the anti-segmentation features can be regarded as the global security features as it makes the Captcha complex on the whole.
In contrary, the anti-recognition features belong to local features. It can defense against character recognition by changing the font style of Captchas.
We list the two fratures as follows:

\noindent \textbf{Anti-segmentation features} \textbf{1. Sophisticated Noisy Background} Try to make the background have little difference with the text for confusing the solver. \textbf{2. Connection Lines} Add extra lines on the text to prevent solver from automatic finding the single character.
\textbf{3. Overlapping} Decreasing the space between the adjacent characters to make them collapsed.

\noindent \textbf{Anti-recognition features} \textbf{1. Character Set} Which charset the text-based Captchas scheme uses. Some schemes only include English letters while some schemes consist of both English letters and Arabic numerals. \textbf{2. Font Style} Using multiple font styles such as solid font, hollow font or font compromise of dots. \textbf{3. Font Size} Using random font size when generating Captcha. \textbf{4. Font Color} Using variable font colors. \textbf{5. Distortion} Distorting characters using attractor fields. \textbf{6. Rotating} Rotating characters with random angles. \textbf{7. Waving} Distorting the characters in a wave fashion.

\subsubsection{Previous Captcha Scheme}
According to the~\cite{Gao2016A}, the previous text-based Captchas can be classified into three categories: character isolated Captcha (Figure~\ref{fig:text-based captchas} (a)), hollow character Captcha (Figure~\ref{fig:text-based captchas} (c)) and Connecting Characters Together (CCT) Captcha (Figure~\ref{fig:text-based captchas} (b) and (d)) based on font style and the space between adjacent characters. Obviously, most of previous text-based Captchas use a single anti-segmentation or anti-recognition features. For examples, Figure~\ref{fig:text-based captchas} (a) only uses one single anti-recognition feature (slight rotating) and Figure~\ref{fig:text-based captchas} (b) uses one anti-segmentation feature (character overlapping) and one anti-recognition feature (character waving).
Of course, some Captcha schemes use a simple combination of the above security features such as Figure~\ref{fig:text-based captchas} (f) uses rotating, background and connection lines. Although using multiple anti-segmentation and anti-recognition features, the text have the same font style or obvious difference comparing to the background.

Thus, the robustness of text-based Captchas is an active topic in the research communities. Researchers have comprehensively evaluated the security of text-based Captchas~\cite{Bursztein2011Text,Bursztein2014The,Gao2016A}.
They demonstrate that text-based Captchas are vulnerable under their attacks as they are able to effectively segment each character. Although the security of text-based Captcha scheme are suspicious by research communities, such scheme is still primary authentication mechanism and are widely deployed most mainstream websites. This wide-spread usage, we think, is due to the following reasons:
(1) the written text on Captcha image may be presented in many styles such as distorted, rotated, hollow, or overlapping characters, complex noisy background or noise interference.
Each of these styles can be designed more complex than previous. For examples, the value of character's rotating angle can be set more large larger than previous, and the background can be became more complex.
(2) The conbination of these styles is able to increase the security strength of Captchas so that it can defeat against current attacks. Current Captchas (Figure~\ref{fig:text-based captchas} (2)) combine many anti-segmentation and anti-recognition features than previous Captchas (See Figure~\ref{fig:text-based captchas} (1)). This increases the security of text-based Captchas. Recent studies suggests that text-based Captchas are still a security mechanism if they are properly designed~\cite{Thomas2013Trafficking,Bursztein2014Easy}. In the following, we will introduce the current Captcha scheme.

\subsubsection{Current Captcha Scheme}
Current Captcha scheme combines many anti-segmentation and anti-recognition features. This makes current text-based Captchas more complex than previous Captcha schemes. Figure~\ref{fig:text-based captchas} (2) shows some representative current Captcha schemes. Obviously, it presents that each current Captcha scheme is compromised of many security features such as different font style, character overlapping, rotating and distorting, connecting lines and so on. Take Figure~\ref{fig:text-based captchas} (k) for an example, this Captcha using two anti-segmentation features (complex background and connecting lines) and five anti-recognition features (Character Set: both English letters and Arabic numerals, Font Style: both solid font and hollow font, Font Size: different font size, Font Color: different font color, Rotating: rotating character with different angles.). Likewise, other Captchas presented in Figure~\ref{fig:text-based captchas} also uses many anti-segmentation and anti-recognition features.

Previous attacking methods on Captchas cannot directly crack current complex text-based Captchas.
The pixel counting approach proposed by Yan \emph{et al.}~\cite{Yan2007Breaking} will lose their efficiency in current Captchas. This is because the character sizes of current Captchas are different, which leads to the number of pixels is not fixed.
Color Filling Segmentation (CFS) method proposed by Yan and used by Gao~\cite{Yan2008A,Gao2013The} are also cast into the shade as sometimes the characters on current Captcha are overlapped, and CFS method cannot segment the connected character.
The powerful generic attacks which claim to crack a wide range of text-based Captchas have been proposed~\cite{Bursztein2014The,Gao2016A}. The attack described in~\cite{Bursztein2014The} is a state-of-the-art way to segment the characters. However, this approach cannot effectively segment the characters when the space between adjacent characters is -3 pixel which is more than the space (generally about -4 pixel) between the adjacent characters of current Captcha.
The work in~\cite{Gao2016A} exploits Gabor filters to extract character components, and then rank the adjacent character components to recombine the individual characters. The major issue of this method is the high error ranking rate whiling cracking hollow characters. This is because the hollow characters will be extracted many possible tiny components, which results in many candidate combinations, increasing the error rate.
Thus, comparing to previous Captchas, current Captcha scheme deployed by many popular web sites are more robustness, and can defeat against previous attacks.

In terms of the above complex Captcha scheme, this paper present a novel generic attack method using deep learning technology.
Unlike previous attacks, our method is able to effectively clean up the complex background and connecting lines, and then the space between adjacent overlapping characters can be expanded as much as possible. At last, we explore to translate the distorted character to a regular one for the purpose of improve the success rate using CNN recognition engine.
Our work aims to call the community to revisit the security of text-based Captchas under the age of Artificial Intelligence.

\begin{table*}[t]
    \centering
    \caption{Mappings from security features to its corresponding numbers}
    \scriptsize
    \label{table: feature_number}
    \begin{tabular}{l|ccc|ccccccc}
        \toprule
        & \multicolumn{3}{|c|}{\textbf{Anti-segmentation Features}} & \multicolumn{7}{|c}{\textbf{Anti-recognition Features}} \\
        \multirow{-2}{*}{\textbf{Security Features}} & Complex Background & Connection Lines & Overlapping & Character Set & Font Style & Font Size & Font Color & Distortion & Rotating & Waving \\
        \midrule
        \textbf{Feature Number} & \circled{0} & \circled{1} & \circled{2} & \circled{3} & \circled{4} & \circled{5} & \circled{6} & \circled{7} & \circled{8} & \circled{9} \\
        \bottomrule
    \end{tabular}
\end{table*}

\subsection{Generative Adversarial Networks} \label{section: GANs}

Generative Adversarial Networks (GANs) is a special type of Artificial Neural Network (ANN), and it was first proposed by Goodfellow in 2014~\cite{Goodfellow2014Generative}. GANs consists of two neural networks competing with each other: one is the generative model \emph{G} that generates the data which similar to the true data; the other is the discriminative model \emph{D} which validate whether the inputting data comes from the training data or \emph{G}. By repeating this competing procedure iteratively, this game will be terminated if the discriminative model \emph{D} cannot identify the inputting data comes from training data or the generative model \emph{G}.

The adversarial trait of GANs makes it has been widely used in video predicting~\cite{Walker2017The}, image processing~\cite{pix2pix2016,CycleGAN2017}, natural language processing~\cite{Yu2016SeqGAN,Li2017Adversarial}, code security~\cite{Xu2016Automatically,Liu2016Delving}. Among these fine studies, Phillip \emph{et al.}~\cite{pix2pix2016} proposed an image-to-image translation method named \emph{Pix2Pix}~\cite{Pix2PixCode} using the conditional GANs\footnote{Conditional GANs learn a mapping from observed image with random noise to output image while GANs learn the mapping from only random noise to output image.} (cGANs)~\cite{Mirza2014Conditional}.
In their work, they can translate an image from one style to another.
Differ from typical GANs, cGANs consists of conditional generator and discriminator. Both the conditional generator and discriminator take the observed image as an input image so that cGANs can converge to a stable state.

This work inspired us that can it translate the distorted Captcha image with complex background and overlapping characters to the regular one with non-overlapping characters and without background?
To do this, we design an effectively segment method for Captchas using a variant of \emph{Pix2Pix}~\cite{Pix2PixCode}.
Our method is able to clean up the complex background and connecting lines on Captcha image, expand the space between the adjacent characters and translate the distorted character to the regular one.
The preliminary experiments prove this method is valid. To our best knowledge, this is the first work to attack Captchas using GANs. 
%\section{Captcha Generator} \label{section: Captcha_generator}
A key success factor for deep learning system is large amount of data to train.
Intuitively, the simple yet direct approach in data collecting is to mine the Captchas from target websites and then paid to mark the labels to them artificially. Obviously, it is a time- or financial-consuming work with a high error rate during marking labels. Further, some type of Captchas are difficult to mine as the corresponding websites limit the accessing frequency.
In addition, there is never such a public Captcha generator which can produce enough different styles of Captchas that match the requirement of the training process.
Thus, the first concern of our work is to develop a flexible, well-calibrated training data generator.

To achieve it, we design an generation model to imitate Captchas production process, and automatically generate different styles of Captchas that are similar to those deployed in real-world websites.
Given a unique text-based Captcha scheme $x$, we first manually analyze the number of the characters and their corresponding security features, $S_{1:N}$, with $N$ parameters such as font style, size and color, rotating, distortion and waving \emph{et al.} described in Section~\ref{section: sccturity_features}.
Here we ranked the from No.1 to 10 shown as Table~\ref{table: feature_number}.
We specifics this analysis results as $\{ M, S_{1:N} \}$, where $M$ and $N$ respectively are the number of characters and its styles.
Then given the content of Captcha, our generation model is able to automatically produce Captchas image $y$, which accords with the above analyzed style.
We define our generation model as follows:
\begin{equation}\label{equation: generator_model}
  y \mid x \sim G_s(x),    x = \{M, N, L_{1:M}, S_{1:N} \}
\end{equation}

Where $G_s(x)$ is the generation model parameterized by unique style $s$, that can generate the Captcha image $y$ based on the unique Captcha scheme $x$. $M$ is the number of characters on the Captcha image and $L_{1:M}$ represents the content of the characters. For each character, our generation model individually generates corresponding style defined by $S_{1:N}$, because each character on some Captchas may have different style.

\textbf{Example:} We use the Captcha image depicted in Figure~\ref{fig:overview} as an example to describe our Captcha generator. This Captcha scheme consists of both English letters and Arabic numerals and the number of characters are fixed (4 characters). The content of the Captcha is $\{7, j, R, U\}$.
It uses 2 anti-segmentation features (Complex Background and Connection Lines) and 4 anti-recognition features (Character Set, Font size, Rotating and Distortion). So the collection of security features number is \{\circled{\small 0}, \circled{\small 1}, \circled{\small 3}, \circled{\small 4}, \circled{\small 6}, \circled{\small 7}\}.
Considering these parameters, the variance $x$ is $\{4, 6, \{7, j, R, U\}, \{\circled{\small 0}, \circled{\small 1}, \circled{\small 3}, \circled{\small 4}, \circled{\small 6}, \circled{\small 7}\}\}$.
For each character in the collection $\{4, 6, \{7, j, R, U\}$, our generator product the subgraph correspond to its security features. At last, the generator aggregates the subgraph to produce a Captcha image.


\section{Overview}
Our work develops an end to end approach for cracking text-based captchas. Instead of relying a large volume of captcha images (which are
increasingly harder to obtain) to train individual models for segmentation and recognition, our approach takes in a much smaller number of
real captchas and it automatically learns how to recognize the target captcha system.
To do so, we develop an approach based on \GANs that learns how to synthesise captchas and then uses the synthesized examples to bulid a base model. We then employ transfer learning to produce a fine tune model quicker when targeting the real captchas. The properties of the images that are abstracted by the former layers of our neural networks are mostly independent of the specific captcha system. This means that one can reuse these parts of the network across similar captcha systems, and, in the process, learning is speeded up considerably. Figure~\ref{fig:overview} depicts the four steps of our attacking system:

\noindent \textbf{\emph{Step 1. Synthesize Captchas:}}
The first step of our attack is to generate enough training captchas that similar to the real ones (Figure~\ref{fig:overview} (1)). To do so, we developed a captcha generator based on \GANs that can automatically generate the similar captchas as the target one. The generator consists of two parts: a synthesizer that can imitate captcha production process, and automatically synthesizes the captchas; a discriminator that can evaluate the similarity between the synthesized sample and real captcha. This process will be terminated until the similarity reaches the threshold. To speed up the convergence of the synthesizer, we use the grid search method~\cite{Audet2006Mesh} to select the optimal parameters that are used to synthesize captchas.

\noindent \textbf{\emph{Step 2. Remove Noise Background and Unify the Font Style:}}
To improve the security of captchas, some websites use obscured background (Figure~\ref{fig:text-based captchas} (e) and (f)) and some use compounded font styles (Figure~\ref{fig:text-based captchas} (g)). Thus, to these captchas, a deep learning model needs to be trained for removing the background or for unifying the font style (Figure~\ref{fig:overview} (2)). This step outputs the clean captcha\footnote{The clean captcha refers to the captcha without noise background or the one with uniform font style.}.
Note that for each captcha scheme, this algorithm needs to train a unique model for addressing the challenge.

\noindent \textbf{\emph{Step 3. Train the Base Model:}}
This step aims to build the base model using the clean captchas for further processing (Figure~\ref{fig:overview} (3)). We use a radical convolutional neural network as our training engine to construct the base model. The training progress will be terminated when the loss of objective function drops to a threshold or when the number of epoches reaches a threshold.

\noindent \textbf{\emph{Step 4. Train the Fine Tune Model:}}
In this final step, we use transfer learning method to fine tune the base model and get the fine tune model (Figure~\ref{fig:overview} (4)). To do so, we retain the former layers of the base model and retrain the later layers by using a small number of real captchas.


%After that, we need to uniform the font style of characters for further processing. This is because on some captcha schemes, the characters have different font style (see Figure~\ref{fig:fill_color} (a), it includes both hollow and solid characters). We have shown that our approach can automatically translate the hollow character to the solid one (Figure~\ref{fig:fill_color}).

%\FIXME{Bla… here we emphasizes that we develop an
%approach based on \GANs that learns how to synthesize captchas and then uses the synthesized examples to build an accurate model.}


%By employing \emph{transfer learning}, our approach is able to produce a new model quicker when targeting a new captcha system. The
%properties of the images that are abstracted by the hidden layers of our neural networks are mostly independent of the specific captcha
%system. This means that one can reuse these parts of the network across captcha systems, and, in the process, learning is speeded up
%considerably.



%
%This section gives an overview of our attacking system which exploits deep learning techniques to recognize current captchas. The system takes in a deformed text-based Captcha with complex noisy background. It automatically generated the corresponding regular Captcha. Then, the regular captcha is cracked by CNN recognition engine. Figure~\ref{fig:overview} depicts the overview of our attack:
%
%\noindent \textbf{\emph{Step 1. Captcha Generator and Preprocessing}}
%
%The first step of our attack is to generate enough training Captchas that similar to the real ones. To do so, we developed a captcha generator that can imitate captcha production process, and automatically generate different styles of captchas that are similar to those deployed in real-world websites. After that, we need to uniform the font style of characters for further processing. This is because on some captcha schemes, the characters have different font style (see Figure~\ref{fig:fill_color} (a), it includes both hollow and solid characters). We have shown that our approach can automatically translate the hollow character to the solid one (Figure~\ref{fig:fill_color}).
%
%%The attack begins from inputting the distorted Captchas with complex background and overlapping characters. Given the characters on Captcha have different styles (see Figure~\ref{fig:fill_color} (a), it includes both hollow and solid characters), it is necessary to uniform the style of these characters for further processing. We have shown that the system can automatically translate the hollow character to the solid one (Figure~\ref{fig:fill_color}).
%
%\noindent \textbf{\emph{Step 2. Generate Regular Captcha}}
%
%Once the style of character on captcha image is uniformed, a deep learning algorithm will be applied to generate the regular captcha shown as Figure~\ref{fig:overview}. We achieve this through hierarchical approaches described as the following three steps:
%
%\noindent \circling{\textcolor{white}{1}} \textbf{Remove Noisy Background:}  To eliminate the influence of complex background on generating regular captcha, the first step is to remove the complex background from the captcha, and produce the captcha with clean background.
%To do so, a deep learning algorithm will be applied to remove the complex noisy background stay on captcha.
%For each captcha scheme, this algorithm needs to train the model for cleaning out the background.
%
%\noindent \circling{\textcolor{white}{2}} \textbf{Expand Space Between Adjacent Characters:} This step aims to increase the distance between two adjacent characters on captcha. We use the same algorithm as the method of removing background to enlarge the inter-character distance and generate a new captcha. Keep in mind that the captcha generated at this stage is distorted.
%
%\noindent \circling{\textcolor{white}{3}} \textbf{Regularization:} In this step, our system is able to automatically translate the distorted captcha to a regular one.
%
%\noindent \textbf{\emph{Step 3. Recognize Captcha}}
%
%In this final step, we use a radical CNN framework as our recognition engine to identify the content of the regular captcha translated from last step.

    \begin{algorithm}[!t]
        \centering
        \caption{Unifying the character font style}
        \label{alg:unify_font_style}
        \begin{algorithmic}[1]
            \REQUIRE~~\\
                $CI$: Captcha Image  \\
                $numChar$: The number of characters on Captchas\\
            \ENSURE~~\\
                $UC$: Captcha with the same font style \\
            \STATE $grayCI \leftarrow getBinaryImage(CI)$ \\
            \STATE $positions[] \leftarrow getHollowPositions(grayCI)$ \\
            \STATE $LEN \leftarrow getPositonsLen(positions[], numChar)$ \\
            \STATE $meanThick \leftarrow getMeanCharThick(CI, positions[])$ \\
            \FOR{$i=1:LEN$}
                \STATE $cFI \leftarrow FillRedColor(CI, positions[i])$ \\
                \STATE $cCI \leftarrow ChangeFillColor(cFI, positions[i])$ \\
                \STATE $charThick \leftarrow getCharThick(cCI, positions[i])$ \\
                \IF{$charThick>meanThick$}
                    \STATE $UC=cropChar(cCI, positions[i], meanThick)$ \\
                \ENDIF
            \ENDFOR
        \end{algorithmic}
    \end{algorithm}
\section{Implementation Details}
\section{Captcha Generator} \label{section: Captcha_generator}
A key success factor for deep learning system is large amount of data to train.
Intuitively, the simple yet direct approach in data collecting is to mine the Captchas from target websites and then paid to mark the labels to them artificially. Obviously, it is a time- or financial-consuming work with a high error rate during marking labels. Further, some type of Captchas are difficult to mine as the corresponding websites limit the accessing frequency.
In addition, there is never such a public Captcha generator which can produce enough different styles of Captchas that match the requirement of the training process.
Thus, the first concern of our work is to develop a flexible, well-calibrated training data generator.

To achieve it, we design an generation model to imitate Captchas production process, and automatically generate different styles of Captchas that are similar to those deployed in real-world websites.
Given a unique text-based Captcha scheme $x$, we first manually analyze the number of the characters and their corresponding security features, $S_{1:N}$, with $N$ parameters such as font style, size and color, rotating, distortion and waving \emph{et al.} described in Section~\ref{section: sccturity_features}.
Here we ranked the from No.1 to 10 shown as Table~\ref{table: feature_number}.
We specifics this analysis results as $\{ M, S_{1:N} \}$, where $M$ and $N$ respectively are the number of characters and its styles.
Then given the content of Captcha, our generation model is able to automatically produce Captchas image $y$, which accords with the above analyzed style.
We define our generation model as follows:
\begin{equation}\label{equation: generator_model}
  y \mid x \sim G_s(x),    x = \{M, N, L_{1:M}, S_{1:N} \}
\end{equation}

Where $G_s(x)$ is the generation model parameterized by unique style $s$, that can generate the Captcha image $y$ based on the unique Captcha scheme $x$. $M$ is the number of characters on the Captcha image and $L_{1:M}$ represents the content of the characters. For each character, our generation model individually generates corresponding style defined by $S_{1:N}$, because each character on some Captchas may have different style.

\textbf{Example:} We use the Captcha image depicted in Figure~\ref{fig:overview} as an example to describe our Captcha generator. This Captcha scheme consists of both English letters and Arabic numerals and the number of characters are fixed (4 characters). The content of the Captcha is $\{7, j, R, U\}$.
It uses 2 anti-segmentation features (Complex Background and Connection Lines) and 4 anti-recognition features (Character Set, Font size, Rotating and Distortion). So the collection of security features number is \{\circled{\small 0}, \circled{\small 1}, \circled{\small 3}, \circled{\small 4}, \circled{\small 6}, \circled{\small 7}\}.
Considering these parameters, the variance $x$ is $\{4, 6, \{7, j, R, U\}, \{\circled{\small 0}, \circled{\small 1}, \circled{\small 3}, \circled{\small 4}, \circled{\small 6}, \circled{\small 7}\}\}$.
For each character in the collection $\{4, 6, \{7, j, R, U\}$, our generator product the subgraph correspond to its security features. At last, the generator aggregates the subgraph to produce a Captcha image.



\begin{figure}
  \centering
  \includegraphics[width=0.5\textwidth]{fig/fill_color.pdf}
  \caption{The procedure of Captcha preprocessing.}
  \label{fig:fill_color}
\end{figure}

\begin{figure}
  \centering
  \includegraphics[width=0.45\textwidth]{fig/captcha_preprocessing/preprocessing.pdf}
  \caption{This figure shows how to unify the font style using the algorithm of morphological dilation.}
  \label{fig: preprocessing}
\end{figure}

\subsubsection{Captcha Preprocessing}
Given some current text-based Captcha schemes mainly consist of both solid characters and hollow characters (see Figure~\ref{fig:text-based captchas} (j) and (k)), the first step of our attack is to unify the different font styles to a fixed style. Here we aims to fill the hollow character with solid core due to the following two reasons:
(1) the hollow character can be easily transformed to the solid one but not the opposite.
(2) the solid character can be extracted more stable features than hollow character at the following step according to our preliminary experiments.

To do so, we first convert the colorful image to black-and-white using the standard threshold selection method proposed by Otsu~\cite{Ostu1979A}\footnote{Note that we only use the binarized image to locate the position of the hollow part of the character other than furthering processing.}.
Then the Color Filling Segmentation (CFS)~\cite{Yan2008A} is used to fill the hollow character with the red color (see Figure~\ref{fig:fill_color} (b)). Next, the color-filled image should be convert to color-coincident image by changing the red filled area to the original character color (see Figure~\ref{fig:fill_color} (c)). At last, the thick filled characters on the color-coincident image need to be cropped as the original character (see Figure~\ref{fig:fill_color} (d)).

However, how to translate the color-coincident image to the target image (shown in Figure~\ref{fig:fill_color}) is a challenge. Fugure~\ref{fig: preprocessing} presents our approach to address this challenge. We first segment the characters using CFS method~\cite{Yan2008A}.  Then we extract the color-filled characters and using morphological algorithm to dilate these characters. To effectively dilate both the edge and inflexion areas, we use the ellipse and cross kernels. We set the size of the kernel to (10, 10), an empirical setting that makes dilated characters clearly visible. After that, we combine these segmented characters to a Captcha image.

Our method for unifying the character font style is described in Algorithm~\ref{alg:unify_font_style}.
The input to the algorithm is the original Captcha image with different font styles and the number of characters on this Captcha, and the output of the algorithm is the Captcha with solid characters. To locate the hollow characters, we first convert the colorful original image to binarized image (line 1). According to the binarized image, we can get the positions of the each hollow characters on the colorful Captcha image (line 2) because the size of colorful image is the same as the binarized image. For each hollow character, we use CFS method described above to fill it with red color (line 6) and then replace the red filled color with the character color to get the color-coincident image (line 7). Finally, we crop the filled character on the color-coincident image to unify the font style of characters (line 10).

\begin{figure}
  \centering
  \includegraphics[width=0.3\textwidth]{fig/loss_anlysis/captcha_generated.pdf} \\
  \caption{Captchas generated when using L1 and L2 loss function.}
  \label{fig: L1_L2}
\end{figure}

\begin{figure}
  \centering
  \subfigure{
      \begin{minipage}[t]{0.22\textwidth}
      \includegraphics[width=\textwidth]{fig/loss_anlysis/loss_L1.pdf}\\
      \center L1 loss curve
      \end{minipage}
  }
  \subfigure{
      \begin{minipage}[t]{0.22\textwidth}
      \includegraphics[width=\textwidth]{fig/loss_anlysis/loss_L2.pdf}\\
      \center L2 loss curve
      \end{minipage}
  }
  \caption{This figure shows the L1 and L2 loss curve when train using Captcha scheme in Figure~\ref{fig: L1_L2}.}
  \label{fig: loss_anlysis}
\end{figure}

\subsection{Generate Regular Captcha}
After unifying the character font style, we need to generate regular Captchas for further recognizing. We achieve this by employing an image-to-image translation algorithm called~\emph{Pix2Pix}~\cite{Pix2PixCode}. This algorithm automatically translate the image from the original style to the target style. In our case, the images to be translated are the Captchas with complex noisy background, overlapping and distorted characters (original style). These are supplied to the algorithm by a Captcha generator developed using a simple script (Section~\ref{section: Captcha_generator}). The algorithm tries to generate regular Captcha with appropriate character spacing and clean background (target style).
Note that if the Captcha to be translated has no background as Google scheme (Figure~\ref{fig:text-based captchas} (l)), our approach just enlarge the space of adjacent character and Regularization operations. Likewise, if the Captcha only contains deformed but no overlapping characters (Figure~\ref{fig:text-based captchas} (j)), our approach only need to regular the characters.

\noindent \textbf{Hierarchical Methods.} In order to generate regular Captcha, we propose a hierarchical method that employ a variant approach of \emph{Pix2Pix} to complete the progressive tasks.
The hierarchical methods are comprised of three sequenced models, and they can respectively achieve three different tasks: removing complex background, expanding space of adjacent overlapping characters and making the distorted character to be a regular one. These three sequence models share the same methodology of~\emph{Pix2pix}.
The main differences of these sequences models are the input and output.
The output of the previous model is the input of the following model.
Take the Captcha in Figure~\ref{fig:overview} for an example, the first model is applied to eliminate the complex background and Connecting lines and output the Captcha with clean background. Then the space of the characters on the output Captcha is expanded by the second model, and produces the Captcha only with distorted characters. Finally, the distorted characters are translated to the regular ones using the last model.
%The inputs of the first step are the images with complex background, overlapping and distorted characters.
%These inputting images are eliminated the complex background by the variant approach which outputs the Captcha images with clean background, overlapping and distorted characters.
%According to the outputs, the second step is to expand the space between the adjacent overlapping characters using the same variant method, and produces the Captcha images only with distorted characters.
%Finally, the distorted characters are translated to the regular ones using the variant method.

\noindent \textbf{Key Algorithm.} The key part of the hierarchical methods is a variant of \emph{Pix2Pix}. It consists of an image generator and an discriminator. They are competing with each other until reach Nash equilibrium during training processing (Section~\ref{section: GANs}).
We take the generation model of removing complex background for detailed introducing our key algorithm.
The goal is to train a generative model that can translate the Captcha image with complex background to the images with clean background.
In order to train this generation model, the input data are the image pairs including the Captcha images with background and the images with clean background.
During training, the image generator produces the fake image that similar to the image with white background by randomly adding the noisy points to it. The fake image and the image with complex background compose an image pair.
For the composed image pair, the discriminator learns to classify between the real and fake pairs. This competing process will be terminated until the discriminator cannot correctly classify the image pairs.
Unlike the \emph{Pix2Pix}, we use the L2 loss function to figure out the loss of the generator because L2 loss function can capturing the overall structure of the Captcha image, which contribute to removing the background and other two tasks (expand the space of adjacent characters and eliminate distorted characters).
\begin{equation}\label{equation: L2_loss}
    \mathcal{L}_{L2}(Gen) = \mathbb{E}_{x,y \epsilon C_{O}, z \epsilon N_{O}} \|y - Gen(x, z)\|_{2}
\end{equation}

Where $x$ is the Captcha image with white background, $y$ is the image with complex background and $z$ is the fake image with the noisy points.

\noindent \textbf{Analysis of Loss Function.} To demonstrate the effectiveness of L2 over L1 loss function, we conduct many perliminary experiments. Take Figure~\ref{fig: L1_L2} (a) for instance, it is expected to be translated to regular Captcha as Figure~\ref{fig: L1_L2} (b). When using L1 loss function, only one character can be successful translated to the target one shown as Figure~\ref{fig: L1_L2} (c). In contrary, all characters can be correctly translated to be the regular ones (Figure~\ref{fig: L1_L2} (d)) when using L2 loss function. This is because the generation model can achieve global optimum when using L2 loss function. Figure~\ref{fig: loss_anlysis} shows that the loss value of L1 has been fluctuating while the value of L2 has been tending to convergence.

%The hierarchical methods are comprised of three sequenced models, and they can respectively achieve the tasks of removing complex background, expanding space between adjacent overlapping characters and translating the distorted Captcha to a regular one. The sequenced models share the same translation model.
%The key part of the translation model is a variant of \emph{Pix2Pix}, which consists of a image generator and discriminator. They are competing with each other until reach Nash equilibrium when training processing (Section~\ref{section: GANs}).
%
%In our case of removing background, the inputting data are the image pairs including the Captcha images with complex background and the images with white background.
%The goal is to train a generative model that can translate the Captcha image with complex background to the images with white background.

\begin{figure}[!t]
  \centering
  \includegraphics[width=0.45\textwidth]{fig/cnn_model.pdf}
  \caption{CNN recognition engine. The input of the recognition recognize is the regular Captcha image, and it output the text of the Captcha image.}
  \label{fig: cnn_model}
\end{figure}

\subsection{Identify Captcha}
In this step, we use a rudimentary CNN framework, \emph{LeNet-5}~\cite{Lecun1998Gradient}, as our recognition engine, to identify the text of the translated Captcha image.
The initial \emph{Lenet-5} was compromised of three convolutional layers, two pooling layers and followed by two fully connected layers.
The convolutional layer extracts the features using a number of filters that are trained during training process. The pooling layer aggregates the features extracted from the convolutional layer for extracting more representative features meanwhile reducing the amount of calculation. The fully connected layer classify the extracted features into target categories. The appropriate number of network layers determines the quality of the extracted features as proper number of layers will extract more representative features.

Unlike the \emph{LeNet-5}, the goal of our recognition engine is to recognize the text of Captchas on the whole, which is more difficult than recognition a single character done by \emph{LeNet-5}. This is because recognizing more characters need to extract more complex and abstract features.
To do so, we redesign the \emph{LeNet-5} and adding another two convolution layers and another three pooling layers.
Figure~\ref{fig: cnn_model} depicts the framework of our recognition engine. Generally, it consists of five convolution layers, five pooling layers and followed by two fully connected layers. Each convolution layer is followed by a pooling layer.

To extract more representative features, each convolutional layer uses a convolution filter of $3 \times 3$ and each pooling layer employs the max-pooling value. Other parameters are the same as the \emph{LeNet-5}.












\begin{table*}
  \centering
  \caption{Target text-based Captcha schemes used in our experiment}
  \label{tab: captcha_show}
  \small
  \begin{tabular}{|c|c|c|c|c|}
    \hline
     &  &  & \multicolumn{2}{|c|}{Security Features} \\
     \cline{4-5}
    \multirow{-2}{*}{Scheme} & \multirow{-2}{*}{Website} & \multirow{-2}{*}{Sample} & Anti-segmentation Features & Anti-recognition Features \\
    \hline
    Google & google.com & \tabincell{c}{\includegraphics[width=0.1\textwidth]{fig/experiment_captchas/google1.jpg} \includegraphics[width=0.1\textwidth]{fig/experiment_captchas/google2.jpg}} & \tabincell{c}{ Overlapping characters, \\ only Enligh letters used} & \tabincell{c}{Varied font size, color, \\ rotating, disortion and waving used} \\
    \hline
    Baidu & baidu.com & \tabincell{c}{\includegraphics[width=0.1\textwidth]{fig/experiment_captchas/baidu3.jpg} \includegraphics[width=0.1\textwidth]{fig/experiment_captchas/baidu2.jpg}} & \tabincell{c}{Connecting Lines, overlapping, \\ only Enligh letters used} & \tabincell{c}{Both hollow and solid characters, \\ varied font size, color, \\ rotating, disortion and waving used} \\
    \hline
    Alipay & alipay.com & \tabincell{c}{\includegraphics[width=0.1\textwidth]{fig/experiment_captchas/alipay1.jpg} \includegraphics[width=0.1\textwidth]{fig/experiment_captchas/alipay2.jpg}} & \tabincell{c}{Overlapping characters used} & \tabincell{c}{Both English letters and \\ Arabic numerals, \\ rotating and distortion used} \\
    \hline
     Wikipedia& wikipedia.org & \tabincell{c}{\includegraphics[width=0.1\textwidth]{fig/experiment_captchas/wikipedia1.png} \includegraphics[width=0.1\textwidth]{fig/experiment_captchas/wikipedia2.png}} & \tabincell{c}{Only English letters used} & \tabincell{c}{Character rotating, \\ distortion and waving} \\
     \hline
     Sohu& sohu.org & \tabincell{c}{\includegraphics[width=0.1\textwidth]{fig/experiment_captchas/sohu1.jpg} \includegraphics[width=0.1\textwidth]{fig/experiment_captchas/sohu2.jpg}} & \tabincell{c}{Complex background, \\ connection lines, \\and overlapping used} & \tabincell{c}{varied font size, color \\ and character rotating} \\
    \hline
    NetEase & 163.com & \tabincell{c}{\includegraphics[width=0.1\textwidth]{fig/experiment_captchas/netease1.jpg} \includegraphics[width=0.1\textwidth]{fig/experiment_captchas/netease2.jpg}} & \tabincell{c}{Complex background, \\ connecting lines} & \tabincell{c}{Both hallow and solid characters, \\ varied rotating angles used} \\
    \hline
    VIPSHOP & vip.com & \tabincell{c}{\includegraphics[width=0.1\textwidth]{fig/experiment_captchas/vipshop1.jpg} \includegraphics[width=0.1\textwidth]{fig/experiment_captchas/vipshop2.jpg}} & \tabincell{c}{Complex background, \\ overlapping characters} & \tabincell{c}{Both English letters \\ and Arabic numerals, \\ varied font colors used} \\
    \hline
     TOUP& tuniu.com & \tabincell{c}{ \includegraphics[width=0.1\textwidth]{fig/experiment_captchas/tuniu1.jpg} \includegraphics[width=0.1\textwidth]{fig/experiment_captchas/tuniu2.jpg}} & \tabincell{c}{Overlapping characters used} & \tabincell{c}{Both English letters \\ and Arabic numerals, \\ waving characters used} \\
    \hline
  \end{tabular}
\end{table*}

\section{Experimental Setup}
\subsection{Data Preparation and Collection}
The captchas used in our evaluation are made up of both training data and testing data, and they come from two different sources. The training captchas are synthesized by our captcha generator (Section~\ref{section: Captcha_generator}) and the testing captchas are collected from target websites using a traditional mining method written through a python script.

\noindent \textbf{Target Websites.} We select the target websites based on the following rules: (1) the captchas of website combines at least one anti-segmentation and three anti-recognition features; (2) the website must have a large number of active users and (3) the target websites should cover as many industries as possible.
%(3) the testing Captchas are relatively easy to mine~\footnote{Some Captcha schemes, such as Tecent Captchas, are hard to mine as each Captcha image has a different URL.}.
According to the three factors, we target 18 famous websites for preparing our data. These target websites can be classified into 5 categories: \emph{E-commerce site} (such as Ebay, Alipay, JD and Alipayexpress), \emph{Social Networks} (such as Live and Sina Microblog), \emph{Search Engine} (such as Google, Baidu, 360 and Bing), \emph{Portal sites} (such as Wikipedia, Sohu and Sina) and other sites (such as Yutube and Microsoft Office).
 Table~\ref{tab: captcha_show} presents some captcha schemes of target websites and their security features.
%\textcolor{red}{5 famous websites for preparing our data: Baidu, Alipay, NetEase, VIPSHOP and TOUP}.


\noindent \textbf{Synthesize Training Captchas.} To ensure the synthesized captcha as likely as the true one, we first manually analyze the traits of the true captchas and aggregates these traits to a feature tuple as described in Section~\ref{section: Captcha_generator}. According to the feature tuple, a large number of training captchas can be synthesized by our generator. In our experiment, we totally synthesized 15 kinds of captchas come from the target websites. For each website, we mined 20000 captchas to be used for training process.

\noindent \textbf{Mine Testing Captchas.} The testing captchas used in our evalustion are automatically mined from the target websites using a python script. For each target website, we apply 2000 captchas to test the efficiency of our attacking system.  We recruited nine participators from our institution to manually mark the labels of the mined captchas. To decrease the marking error rate, the nine participator are divided into three groups and each group has three people. We choose these captcha which two participators mark the correct tags of the captchas at the same time as the testing data.

\subsection{Implementation}
Our prototyped attacking system built upon a variant of \emph{Pix2Pix} framework~\cite{Pix2PixCode} in Tensorflow. The developed software ran on an ordinary
server with a 3.2-GHz Inter Xeon CPU with 100-GB RAM and a TITAN Xp GPU. The operating system is Ubuntu 16.04.

%\subsection{Captchas Show}

\section{Experimental Results}
In this section, we first present the overall success rate for breaking the captchas collected from the target websites. Our results show that the overall success rate ranges from 3\% to 92\% speeding no more than 150 milliseconds.
We then analyze how the number of training captchas effects the success rate.
Finally, we demonstrate the risk of our attacking approach by simulating a real world attacking process before evaluating the robustness of the secure features using various types of text-based captchas.

\begin{table}[t]
    \centering
    \small
    \caption{The overall success rate and average attack speeds for each Captcha scheme.}
    \label{table: current_rate}
    \begin{tabular}{lccc}
        \toprule
        & \multicolumn{2}{c}{Success rate} & \\
        \cline{2-3}
        \multirow{-2}{*}{Scheme} & Base Model & Fine Tune Model  & \multirow{-2}{*}{Speed (ms)}\\
        \midrule
        Google & 0\% & 3\% & 60 \\
        Microsoft & 36.5\% & 69.5\% & 60 \\
        Wikipedia & 6\% & 35.5\% & 60 \\
        Ebay & 50\% & 86.5\% & 50 \\
        Alipay & 26.5\% & 45\% & 60 \\
        Baidu & 7.5\% & 19.5\% & 150 \\
        JD & 60\% & 86\% & 140 \\
        Sina & 44.5\% & 48.5\% & 20 \\
        Weibo & 4.69\% & 27\% & 50 \\
        Qihu360 & 48.5\% & 56\% & 50 \\
        Sohu & 83\% & 92\% & 130 \\
        \bottomrule
    \end{tabular}
\end{table}

\begin{figure}
  \centering
  \subfigure{
    \begin{minipage}[t]{0.45\textwidth}
        \includegraphics[width=0.15\textwidth]{fig/experiment_mnist/0-1.png}
        \includegraphics[width=0.15\textwidth]{fig/experiment_mnist/0-2.png}
        \includegraphics[width=0.15\textwidth]{fig/experiment_mnist/0-7.png}
        \includegraphics[width=0.15\textwidth]{fig/experiment_mnist/0-8.png}
        \includegraphics[width=0.15\textwidth]{fig/experiment_mnist/0-9.png}
        \includegraphics[width=0.15\textwidth]{fig/experiment_mnist/0-10.png}
    \end{minipage}
    }
    \subfigure{
    \begin{minipage}[t]{0.45\textwidth}
        \includegraphics[width=0.15\textwidth]{fig/experiment_mnist/4-1.png}
        \includegraphics[width=0.15\textwidth]{fig/experiment_mnist/4-2.png}
        \includegraphics[width=0.15\textwidth]{fig/experiment_mnist/4-5.png}
        \includegraphics[width=0.15\textwidth]{fig/experiment_mnist/4-6.png}
        \includegraphics[width=0.15\textwidth]{fig/experiment_mnist/4-7.png}
        \includegraphics[width=0.15\textwidth]{fig/experiment_mnist/4-8.png}
    \end{minipage}
    }
    \subfigure{
    \begin{minipage}[t]{0.45\textwidth}
        \includegraphics[width=0.15\textwidth]{fig/experiment_mnist/8-9.png}
        \includegraphics[width=0.15\textwidth]{fig/experiment_mnist/8-10.png}
        \includegraphics[width=0.15\textwidth]{fig/experiment_mnist/8-3.png}
        \includegraphics[width=0.15\textwidth]{fig/experiment_mnist/8-4.png}
        \includegraphics[width=0.15\textwidth]{fig/experiment_mnist/8-5.png}
        \includegraphics[width=0.15\textwidth]{fig/experiment_mnist/8-6.png} \\
        \centering (a) MNIST dataset
    \end{minipage}
    }
    \subfigure{
    \begin{minipage}[t]{0.45\textwidth}
        \includegraphics[width=0.15\textwidth]{fig/experiment_onechar/0-1.png}
        \includegraphics[width=0.15\textwidth]{fig/experiment_onechar/0-2.png}
        \includegraphics[width=0.15\textwidth]{fig/experiment_onechar/0-7.png}
        \includegraphics[width=0.15\textwidth]{fig/experiment_onechar/0-8.png}
        \includegraphics[width=0.15\textwidth]{fig/experiment_onechar/0-9.png}
        \includegraphics[width=0.15\textwidth]{fig/experiment_onechar/0-10.png}
    \end{minipage}
    }
    \subfigure{
    \begin{minipage}[t]{0.45\textwidth}
        \includegraphics[width=0.15\textwidth]{fig/experiment_onechar/D-1.png}
        \includegraphics[width=0.15\textwidth]{fig/experiment_onechar/D-2.png}
        \includegraphics[width=0.15\textwidth]{fig/experiment_onechar/D-5.png}
        \includegraphics[width=0.15\textwidth]{fig/experiment_onechar/D-6.png}
        \includegraphics[width=0.15\textwidth]{fig/experiment_onechar/D-7.png}
        \includegraphics[width=0.15\textwidth]{fig/experiment_onechar/D-8.png}
    \end{minipage}
    }
    \subfigure{
    \begin{minipage}[t]{0.45\textwidth}
        \includegraphics[width=0.15\textwidth]{fig/experiment_onechar/o-1.png}
        \includegraphics[width=0.15\textwidth]{fig/experiment_onechar/o-2.png}
        \includegraphics[width=0.15\textwidth]{fig/experiment_onechar/o-7.png}
        \includegraphics[width=0.15\textwidth]{fig/experiment_onechar/o-8.png}
        \includegraphics[width=0.15\textwidth]{fig/experiment_onechar/o-13.png}
        \includegraphics[width=0.15\textwidth]{fig/experiment_onechar/o-14.png} \\
        \centering (b) Single character dataset
    \end{minipage}
    }
  \caption{This figure shows some examples of original samples and their regular one transformed by our approach.}
  \label{fig: mnist_show}
\end{figure}

\subsection{Overall Success Rate}
In this experiment, the training captchas were synthesized by our captcha synthesizer. For each captcha scheme, the number of training data is 200k. The corresponding testing data were collected from the target websites and its number is 2000.
\subsubsection{Evaluation using current text-based captchas}
Table~\ref{table: current_rate} shows the cracking success rate before or after model transferring and the cracking speed. Using the base model, our approach can crack all captchas except the Google captcha. Furthermore, the cracking success rate improves significantly using fine tune model that is built using only 500 real captchas. We failed to crack google captcha using the base model because google employed many fonts and multiple deformation characteristics to generate the captcha so that our synthesizer failed to learn the generation parameters of the real examples. But even so, our approach is still able to crack 3\% google captchas after model transferring based on the base model. For Microsoft, Wikipedia, Ebay, Alipay, Baidu and Weibo captchas, the success rates before model transferring respectively are 36.5\%, 6\%, 50\%, 26.5\%, 7.5\% and 4.69\%. After model transferring, the success rates significantly increase and they are 69.5\%, 35.5\%, 86.5\%, 45\%, 19.5\% and 27\%, respectively. This illustrates that our model transferring approach is effective for cracking captchas. For other captcha schemes such as JD, Sina, Qihu360 and Sohu, they are easier to be cracked by our approach. This is proved that the success rates are more than 45\% using fine tune model. In general, our cracking success rate is far more than 1\%, the secure threshold at which the captcha is considered ineffective~\cite{Bursztein2011Text}. On average, our approach takes less than 150 milliseconds per cracking a captcha. Specifically, it takes no more than 60 milliseconds to crack a captcha without background. This means our approach poses a great threat to current widely used text-based captchas.

\begin{table*}[t]
    \centering
    \caption{The success rate for cracking previous captchas. "reCaptcha(11)" denotes the reCaptcha at 2011.}
    \label{table: previous_rate}
    \scriptsize
    \begin{tabular}{|c|c|c|c|c|c|c|c|c|c|c|c|}
        \hline
        & \multicolumn{2}{|c|}{Success rate}& & \multicolumn{2}{|c|}{Success rate} & &\multicolumn{2}{|c|}{Success rate} & & \multicolumn{2}{|c|}{Success rate}\\
        \cline{2-3} \cline{5-6} \cline{8-9} \cline{11-12}
        \multirow{-2}{*}{Scheme} & ref.~\cite{Bursztein2011Text} & ours & \multirow{-2}{*}{Scheme} & ref.~\cite{Gao2016A} & ours & \multirow{-2}{*}{Scheme} & ref.~\cite{Bursztein2014The} & ours & \multirow{-2}{*}{Scheme} & ref.~\cite{George2017A} & ours \\
        \hline
        Wikipedia & 25\% & 35.5\% & Wikipedia & 23.8\% & 35.5\% & Wikipedia & 28.29\% & 35.5\% & reCaptcha(11) & 66.6\% & 87.5\% \\
        \hline
        eBay & 43\% & 86.5\% & eBay & 58.8\% & 86.5\% & eBay & 51.39\% & 86.5\% & Yahoo!(16) & 57.4\% & 63\% \\
        \hline
        Captcha.net & 73\% & 99.5\% & reCaptcha(11) & 77.2\% & 87.5\% & CNN & 51.09\% &  & PayPal & 57.1\% & \\
        \hline
        Digg & 20\% & 95\% & Yahoo!(16) & 5.2\% & 63\% & reCaptcha(11) & 22.67\% & 87.5\% & MNIST & 97.89\% & 97.45\% \\
        \hline
        Blizzaer & 70\% & 100\% & Baidu(16) & 46.6\% & 97.5\% & reCaptcha(13) & 22.34\% & 90\% & & &\\
        \hline
        Megaupload & 93\% & 100\% & Microsoft & 16.2\% & 72.5\% & Yahoo!(14) & 5.33\% &  & & & \\
        \hline
        NIH & 72\% & 99\% & Amazon & 25.8\% & 79\% & Baidu(11) & 38.68\% & 83.5\% & & & \\
        \hline
        Reddit & 42\% & 98\% & Taobao & 23.4\% & 91.5\% & Baidu(13) & 55.22\% & 89\% & & &  \\
        \hline
        Slashdot & 35\% & 86.5\% & Sina & 9.4\% & 90.5\% & & & & & & \\
        \hline
        Authorize & 66\% & 100\% & QQ & 56\% & 94.5\% & & & & & & \\
        \hline
    \end{tabular}
\end{table*}

\subsubsection{Evaluating using Previous text-based captchas}
In this section, we compare our approach with previous works~\cite{Bursztein2011Text,Gao2016A,Bursztein2014The,George2017A} using previous text-based captchas. Since many websites have update their captcha schemes except Wikipedia and Ebay, we collected the previous captchas using the following three principles: (1) we use the online captcha generator such as Captcha.net to produce the true captchas; (2) we employ the open programs that previous papers have provided to generate the true captchas and (3) we collect previous captchas from the public datasets. For each captcha scheme, we totally collected 700 real samples including 500 for model transferring and 200 for testing. We use our captcha synthesizer (Section~\ref{section: captcha_generator}) to generate 200k captchas per captcha scheme for training the base model. Figure 12 depicts some examples of the previous real captchas and the generated versions that synthesized by our generator. Note that our synthesizer aims to generate the same style of captcha as the target captcha, other than the same captcha as the target one. So the generated captcha has some subtle differences from the real one. For example, Figure~\ref{fig: generate_show} (e) shows the real captcha deployed by Microsoft, and it slightly different from the generated one shown in (f) because the each character on captcha image is randomly rotated and distorted. Nevertheless, the generated captchas contain the main features of the original real world captchas.

Table~\ref{table: previous_rate} summaries the success rate for recognizing previous captchas. It shows that our approach is able to solve every captcha scheme with accuracy higher than previous approaches proposed in~\cite{Bursztein2011Text,Gao2016A,Bursztein2014The,George2017A}. Our approach in its best configuration can crack all Blizzaer, Megaupload and Authorize captchas that are evaluated in~\cite{Bursztein2011Text}. Unlike previous segment then recognize method~\cite{Bursztein2011Text,Gao2016A,Bursztein2014The,George2017A}, we proposed an end to end approach that can crack the captcha quicker.

%In this section, we compare our result with previous works~\cite{Bursztein2011Text,Gao2016A,Bursztein2014The,George2017A} using the previous text-based captchas.
%For the previous captchas, the real world testing data are hard to be collected because most websites have updated their Captcha scheme which is more complex than prior one. To acquire the true previous captchas, we first use some online captcha generator such as Captcha.net to produce the true Captcha images. We then employ the generators that previous papers has provided to generate captchas. If neither of the two cases, we use our captcha generator to produce the testing data.
%
%Figure~\ref{fig: generate_show} depicts some captcha examples of the previous real world captchas and our generated versions that produced by our generator (Section~\ref{section: Captcha_generator}). Recall that our captcha generator is able to generate the similar style captchas with the real world captcha scheme. So the generated captcha has some subtle differences from the real one. For example, Figure~\ref{fig: generate_show} (e) shows the real captcha deployed by Microsoft, and it slightly different from the generated one shown in (f) because the each character on captcha image is randomly rotated and distorted.
%Nevertheless, the generated captchas contain the main features of the original real world captchas.


\begin{figure}
  \centering
  \subfigure{
    \begin{minipage}[t]{0.2\textwidth}
        \includegraphics[width=0.9\textwidth]{fig/generate_captchas/reCAPTCHA_prior.png}\\
        \center (a) reCAPTCHA
    \end{minipage}
    }
    \subfigure{
    \begin{minipage}[t]{0.2\textwidth}
        \includegraphics[width=0.9\textwidth]{fig/generate_captchas/reCAPTCHA_generator.png}\\
        \center (b) Generated version
    \end{minipage}
    }
    \subfigure{
    \begin{minipage}[t]{0.2\textwidth}
        \includegraphics[width=0.9\textwidth]{fig/generate_captchas/yahoo_prior.png}\\
        \center (c) Yahoo!
    \end{minipage}
    }
    \subfigure{
    \begin{minipage}[t]{0.2\textwidth}
        \includegraphics[width=0.9\textwidth]{fig/generate_captchas/yahoo_generator.png}\\
        \center (d) Generated version
    \end{minipage}
    }
    \subfigure{
    \begin{minipage}[t]{0.2\textwidth}
        \includegraphics[width=0.9\textwidth]{fig/generate_captchas/microsoft_prior.png}\\
        \center (e) Microsoft
    \end{minipage}
    }
    \subfigure{
    \begin{minipage}[t]{0.2\textwidth}
        \includegraphics[width=0.9\textwidth]{fig/generate_captchas/microsoft_generator.png}\\
        \center (f) Generated version
    \end{minipage}
    }
    \subfigure{
    \begin{minipage}[t]{0.2\textwidth}
        \includegraphics[width=0.9\textwidth]{fig/generate_captchas/amazon_prior.png}\\
        \center (g) Amazon
    \end{minipage}
    }
    \subfigure{
    \begin{minipage}[t]{0.2\textwidth}
        \includegraphics[width=0.9\textwidth]{fig/generate_captchas/amazon_generator.png}\\
        \center (h) Generated version
    \end{minipage}
    }
    \subfigure{
    \begin{minipage}[t]{0.2\textwidth}
        \includegraphics[width=0.9\textwidth]{fig/generate_captchas/baidu_prior.png}\\
        \center (i) Baidu
    \end{minipage}
    }
    \subfigure{
    \begin{minipage}[t]{0.2\textwidth}
        \includegraphics[width=0.9\textwidth]{fig/generate_captchas/baidu_generator.png}\\
        \center (j) Generated version
    \end{minipage}
    }
  \caption{Some examples of the real world captchas and the corresponding generated version. The captchas on the left column are the previous real world schemes and on the right column are their generated version.}
  \label{fig: generate_show}
\end{figure}

\begin{figure}
  \centering
  \includegraphics[width=0.45\textwidth]{fig/training_set.pdf}
  \caption{Impact of the number of training set.}
  \label{fig: training_set}
\end{figure}

\begin{table}[t]
    \centering
    \caption{Classification results on both standard MNIST and generated dataset.}
    \label{table: mnist}
    \begin{tabular}{lccc}
        \toprule
        \multicolumn{4}{c}{MNIST dataset} \\
        \midrule
        \# of per class & 100  & 200 & 500\\
        \midrule
        RCN & 97.89\% & 98.10\% & 98.60\% \\
        Ours & \textbf{91.40\%} & \textbf{94.50\%} & \textbf{98.75\%} \\
        \toprule
        \multicolumn{4}{c}{Single character dataset} \\
        \midrule
        \# of per class & 100  & 200 & 500\\
        \midrule
        RCN & 82.67\% & 81.13\% & 89.92\% \\
        Ours & \textbf{87.68\%} & \textbf{89.84\%} & \textbf{92.29\%} \\
        \bottomrule
    \end{tabular}
\end{table}

\subsubsection{Evaluation using MNIST datasets}
We evaluate our approach using the MNIST datasets, and compared the results with the latest work~\cite{George2017A} which can recognize the MNIST with a high accuracy using a small number of training data. In this experiment, we selected 100, 200 and 500 MNIST examples for each digit for training, respectively.
Table~\ref{table: mnist} demonstrate that our approach is superior to RCN when the total number of training examples more than 5000. Indeed, RCN performs better than ours when using a small number of training data as our approach cannot correctly transformed the similar characters to the regular one.

In addition, we generated single character dataset which contains digits and English characters. Each character in this dataset is rotated and distorted. The stye of the generated dataset is a little like MNIST's style.
We use such dataset to evaluate both RCN and our approach, and the results are shown in Table~\ref{table: mnist}. It obviously show that our approach is able to recognize most characters while RCN is invalid. Figure~\ref{fig: mnist_show} shows some examples of original data and the corresponding regular version.

\subsection{Impact of the number of training data}
We would like to know how the number of training examples affects the success rate of the attack. To do so, we used all the collected captchas and we varied the number of training captchas from 1000 to 100000 for each scheme. Figure~\ref{fig: training_set} shows how the cracking success rate changes as the number of training Captchas increases. When the size of the training examples is 100000, our approach can break 97.5\% VISHOP captchas, 96.6\% Alipay captchas, 80.4\% TOUP Captchas, 79.6\% BetEase captchas and 48\% Baidu captchas respectively. The success rate drops significantly when the number of training captchas is less than 10000. Beyond this point, our approach cannot extract representative features of the captchas. The degradation features makes it difficult for our algorithm to successful recognize the content of the captcha image.
Due to the degradation, our approach can just crack 2.65\% Baidu captchas, 3.45\% NetEase captchas, 4.25\% Alipay captchas, 7.55\% TOUP captchas and 10.4\% VISHOP captchas respectively when the number of training examples is only 1000. Nonetheless, the success rates are higher than 1\%, the security threshold that determines whether or not the captcha is secure.
Generally, our approach can achieve a high success rate when the number of training examples is 20000.

\begin{table*}
    \centering
    \caption{The success rate of each combinations security features}
    \label{table: combinations_resuts}
    \begin{tabular}{|c|c|c|c|c|c|c|}
        \hline
        No. & Sample Captcha & Overlapping & Rotating & Distorting & Waving & Success Rate \\
        \hline
        \tabincell{c}{1} & \tabincell{c}{\includegraphics[width=0.1\textwidth]{fig/combination/combination1.jpg}} & \tabincell{c}{$\surd$} & \tabincell{c}{$\surd$} & & & 74.85\% \\
        \hline
        2 & \tabincell{c}{\includegraphics[width=0.1\textwidth]{fig/combination/combination2.jpg}} & $\surd$ &  & $\surd$ & & 65.05\% \\
        \hline
        3 & \tabincell{c}{\includegraphics[width=0.1\textwidth]{fig/combination/combination3.jpg}} & $\surd$ &  & & $\surd$ & 58.8\%  \\
        \hline
        4 & \tabincell{c}{\includegraphics[width=0.1\textwidth]{fig/combination/combination4.jpg}} &  & $\surd$ & $\surd$ & & 64.95\% \\
        \hline
        5 & \tabincell{c}{\includegraphics[width=0.1\textwidth]{fig/combination/combination5.jpg}} & & $\surd$ & & $\surd$ & 82.35\% \\
        \hline
        6 & \tabincell{c}{\includegraphics[width=0.1\textwidth]{fig/combination/combination6.jpg}} & & & $\surd$ & $\surd$ & 62.45\% \\
        \hline
        7 & \tabincell{c}{\includegraphics[width=0.1\textwidth]{fig/combination/combination7.jpg}} & $\surd$ & $\surd$ & $\surd$ & & 57.50\% \\
        \hline
        8 & \tabincell{c}{\includegraphics[width=0.1\textwidth]{fig/combination/combination8.png}} & $\surd$ & $\surd$ & $\surd$ & $\surd$ & 52.50\% \\
        \hline
        9 & \tabincell{c}{\includegraphics[width=0.1\textwidth]{fig/combination/combination9.png}} & \multicolumn{4}{|c|}{All security features} & 46.30\% \\
        \hline
    \end{tabular}
\end{table*}

\subsection{Impact of Security Features}
Intuitively, the success rate of the attack would be influenced by the number of security features used by the captcha because the complexity of the captcha depends on the number of the security features. To evaluate this effect, we choose the commonly used security feature such as overlapping, rotating, distortion, waving and background for the experiment.
First, we evaluate how a single commonly used security feature effect the success rate. We then evaluate our attack using our synthetic captchas which combines of a number of commonly used security features. Note that the synthetic captchas are more complex than current any real world captcha scheme. At last, we generate a most complicated captcha using all security features to evaluate our approach.

\noindent \emph{1) Evaluation of single security feature}

\textbf{Overlapping.} Overlapping decreases the space between adjacent characters and makes them collapsed, and it is considered the most effective anti-segmentation security feature~\cite{Bursztein2011Text} which is widely deployed by lots of popular websites.

We use the TOUP scheme as a case study to evaluate the security of overlapping. We modify the original TOUP captchas through increasing character overlapping by 4, 6, 8 and 10 pixels\footnote{The overlapping of the original captcha is no more than 3 pixels.}, respectively. For each overlapped captchas, we generated 20000 training examples and 2000 testing examples. We then run our attack on these synthetic overlapped captchas, and our new success rate are 65\%, 50.05\%, 42.55\% and 25.1\%, respectively.  This result demonstrates the more overlapped the characters, the less successful our approach would be. Although the success rate drops significantly when the character overlapping more than 6 pixels, it still pose a real threat to captcha and also superior to human because human cannot recognize the captcha when the character overlapping reaches 6 pixels~\cite{Bursztein2014The}.

\textbf{Rotating.} To evaluate how the character rotating effects the success rate, we choose the Alipay scheme for implementing the experiment as Alipay captcha randomly rotates the character. This is more difficult than other scheme which rotates the character in one single direction. The rotating angle of Alipay is less than 15 degrees through our careful analysis. In this experiment, we increases the random rotating angle to 30 degrees. We use 20000 such captchas for training and 2000 for testing, and our approach can achieve a success rate of 80.5\%, a little lower than the original success rate of 86.15\%. This indicates rotating does not have a positive effect in enhancing the security of captcha under our attack.

\textbf{Distorting.} Character distortion is widely used by most current text-based captcha schemes. It can confuse the bot program for recognizing the similar distorted characters. For example, the character "O" and the number "0" looks very similar when they are distorted. This makes the malicious bots misidentifying "O" as "0" or "0" as "O".

In this experiment, we generate 20000 training captchas which are similar to the Alipay scheme. But the generated captchas are more distorted than Alipay scheme. Using 2000 captchas for testing, our approach is able to crack 67.5\% which is lower than the original success rate of 90.10\%. This indicates that although character distortion enhancing the security of captcha to some extent, our approach can still break such captcha with a high success rate.

\textbf{Waving.} In this evaluation, we use the captcha generator proposed by Tuan~\emph{et al.}~\cite{Le2017Using} to produce 20000 training examples for training. Using the trained model, our approach is able to recognize 1977 out of 2000 testing captchas, and have a success rate of 98.85\%, higher than Tuan's success rate of 93.6\%. We only failed to break 23 captchas because our model is confused by some similar waving characters such as character "O" and number "0", character "1" and "l", \emph{etc}.  This experiment demonstrates that waving transformation of characters does not contribute to enhancing the security of captcha under our attack.


\noindent \emph{2) Evaluation of combining security features}

Combination security features can offer a more secure protection than a single one. In this section, we perform a number of experiments to test various combinations of the security features, and evaluate how each combination effects our attack. We conduct this experiment to search whether or not having a combination can defeat against our attack. We choose four basic security features from which we design eight different combinations and we test all of them. We select Alipay captcha for our experiments, and the size of training and test set is 20000 and 2000, respectively.

Table~\ref{table: combinations_resuts} summaries the experiment results, listing a number of combinations of security features along with its influence on the success rate of our attack.
The results suggest the following insights.  Firstly, the combinations of two or more security features are indeed more secure than just a single one. Secondly, character overlapping or distorting can offer a stronger protection than rotating or waving because overlapping or distorting changes the style of character which indeed enhances security. This can be proved by the results of experiments No. 1 - 6. Lastly, the combination of all security features has a positive effect on defeat against our attack. However, our approach is able to crack 46.30\% such captchas, much higher than 1\% at which captchas is considered ineffective. On the other hand, it will decreases the usability when blindly using more security features. Nonetheless, our attack still can crack all the complex schemes with a high success rate. Therefore, the text-base captcha is no longer a secure scheme to prevent the malicious bot programs.

\section{Related Work}
Our work lies at the intersection between adversarial machine learning and cracking text-based Captchas methods. This works brings together techniques developed in the domain of adversarial deep learning and character recognition to develop the attack on current text-based Captchas.

\noindent \textbf{Optical Character Recognition technology} Naor proposed the definition of Automated Turing Tests for automatically distinguishing humans from computer systems~\cite{Naor1996Verification}. This definition was firstly implemented by Lillibridge \emph{et al.}~\cite{Lillibridge2001Method} who developed the first practical scheme to prevent malicious bot programs from registering web accounts. However, such scheme was soon defeated by common used Optical Character Recognition (OCR) technique. Since then, a various of text-based and corresponding attacks have been proposed.

\noindent \textbf{Attacks on the individual Captchas} Mori~\emph{et al.}~\cite{Mori2003Recognizing} proposed an attack on Gimpy and EZ-Gimpy using complex object recognition algorithm. Their experiments shows that their approach can break 33\% Gimpy and 92\% EZ-Gimpy. Yan~\emph{et al.}~\cite{Yan2007Breaking} used a simple character segmentation method to break most Captchas provided by Captchaservices.org. They segmenting the character by counting the number of pixels of each individual character. Then they improved their character segmentation technique for attacking Microsoft-like Captcha schemes deployed by Yahoo!, Microsoft and Google~\cite{Yan2008A}. Gao~\emph{et al.}~\cite{Gao2013The} firstly proposed a generic method to break the hollow Captcha scheme. They first extracted the hollow character strokes based on color filling segmentation (CFS) method, and then searched for possible combinations of adjacent character strokes to form the individual characters. Recently they break Microsoft’s two-layer Captchas using similar approach with a high success rate~\cite{Gao2017Research}.
The above solutions first need to segment characters and then recognition them. But for the skewed and overlapped letters, their methods are hard to successfully segment as the characters are gathered together. To such scheme, Stark~\emph{et al.}~\cite{Stark2015CAPTCHA} presented an approach that can recognize the Captchas based on Active Deep Learning. Their approach can break Google's reCAPTCHA with a high success rate using a small number of labeled data.

\noindent \textbf{Generic attacks on text-based Captchas} Bursztein~\emph{et al.}~\cite{Bursztein2014The} reported a generic method to break a wide of text-based Captchas in a single step using machine learning algorithm. Their approach can recognize the Captcha by addressing the segmentation and recognition problems simultaneously. It scores all possible ways of character combinations to segment a captcha using reinforcement learning algorithm and search for the best possible one as the result.
More recently, Gao~\emph{et al.}~\cite{Gao2016A} present another generic approach that solved text-based Captchas. They first used Gabor filter to extract character components of the Captcha in four different directions, and then the extracted character components is ranked ordered by a graph search algorithm before recognizing the ranked components using Convolutional Neural Network.
The success of the two above approaches lies in the relative clean background. That is their approaches perfect poor when the Captcha has complex background as VISHOP scheme shown as Figure~\ref{fig:text-based captchas}. This is because their segmentation method cannot distinguish the individual character from such background.
George~\emph{et al.}~\cite{George2017A} proposed a novel network model named RCN that can break text-based Captchs using a samll number of training data.

\noindent \textbf{Study on the robustness of text-based Captchas} Bursztein~\cite{Bursztein2011Text} conducted a systematic study on existing text-based Captchas based on distorted characters. Among studying 15 Captcha deployed on popular websites, they found that 13 schemes excepting for Google and reCAPTCHA are vulnerable to automated attacks. They also explored whether it is easy to recognize the Captchas for human by conducting a large scale evaluation of Captchas from human's perspective~\cite{Bursztein2010How}. They collected 21 most widely used captcha schemes and asked the participants to manually recognize these Captchas. Their experiment results indicated that Captchas general were difficult for humans, especially for the audio Captcha schemes.
Krol~\emph{et al.}~\cite{Krol2016Better} carried out a user study on both reCAPTCHAs and "gamified" Captchas. They found that participants preferred to reCAPTCHAs than gamified Captchas because they were familiar with reCAPTCHAs.

\noindent \textbf{Attacks on other Captcha schemes} In addition to text-based Captchas, there are other Captchas such as image-based Captchas~\cite{Elson2007Asirra, Athanasopoulos2006Enhanced, Areyouhuman, Mohamed2017On, Gossweiler2009What}, audio-based Captchas~\cite{Schlaikjer2010A, Bigham2009Evaluating}.
However, researchers have proposed some attacks on image-based Captchas~\cite{Mohamed2014A, Gao2014An, Sivakorn2016I} and audio-based Captchas~\cite{Tam2008Breaking, Meutzner2014Reducing, Bursztein2009Decaptcha}, respectively. Further, such Captcha schemes are vulnerable to side-channel attacks~\cite{Hernandezcastro2009Side} as the limited number of example Captchas can be easily mined by the adversary.

\noindent \textbf{Adversarial machine learning} Recently, adversarial machine learning~\cite{Huang2011Adversarial} begins to be used in the field of information security. Researchers have found such technique can be used to evading classifiers of malicious software~\cite{Xu2016Automatically, Rosenberg2017Generic}, constructing specific malicious examples bypass intrusion detection system or spam e-mail filtering~\cite{Barreno2006Can} and cheat any classifiers trained by machine learning algorithms~\cite{Goodfellow2015Explaining, Miyato2015Distributional}.


%The Automated Turing Tests were first proposed by Naor~\cite{Naor1996Verification}, but they did not provide a formal definition. Lillibridge \emph{et al.}~\cite{Lillibridge2001Method} developed the first practical Automated Turing Test to prevent bots from automatically registering web pages. This system was effective for a while and then was defeated by common Optical Character Recognition (OCR) technology.
%
%Text-based Captchas based on English letters and Arabic numerals, is still the most widely deployed scheme. Many research communities focus on developing attack approaches for existing text-based Captchas, and then explore guidelines for better designs.
%
%A Captcha is considered broken if it can be automatically solved at a rate above 1\% at which Captchas are considered ineffective~\cite{Bursztein2011Text}. 
\input{conclusion}
%\section{Materials}

1. CPATCHAs are computer-generated tests that humans can pass but current computer systems cannot. CAPTCHAs provide a method for automatically distinguishing a human from a computer program, and therefore can protect Web services from abuse by so-called "bots".

most CAPTCHAs consist of distorted images, usually text, for which a user must provide some description.

CAPTCHAs can distinguish between humans and computers with high probability, they are used for many different security applications: they prevent bots from voting continuously in online polls, automatically registering for millions of spam email accounts, automatically purchasing tickets to buy out an event, etc~\cite{Tam2008Breaking}.

2. We carry out a systematic study of existing visual CAPTCHAs based on distorted characters.

For example, Gmail improves its service by blocking access to automated sapmmers, eBay improves its marketplace by blocking bots from flooding the site with scams, and Facebook limits creation of fraudulent profiles used to spam honest users or cheat at games. The most widely used CAPTCHA schemes use combinations of distorted characters and obfuscation techniques that human can recognize but that may be difficult for automated scripts.

In fact, as we substantiate by thorough study, many popular websites still rely on schemes that are vulnerable to automated attacks.

3. The abuse of the resources of online services using automated means, such as denial-of-service or password dictionary attacks, is a common security problem~\cite{Mohamed2014A}.

4. Since their inception, captchas have been widely used for preventing fraudsters from performing illicit actions.

In this paper, we conduct a comprehensive study of reCaptcha, and explore how the risk analysis process is influenced by each aspect of the request.

1. Captcha is a security mechanism designed to differentiate between computers and humans, and is used to defend against malicious bot programs.

We propose a simple but an effective method to attack the two-layer Captcha deployed by Microsoft, and achieve a success rate of 44.6\% with an average speed of 9.05s on a standard desktop computer (with a 3.3-GHz Intel Core i3 CPU and 2-GB RAM), thus demonstrating clear security issues.

Researchers have recently claimed that their simple generic attacks have broken a wide range of text-based Captchas in a single step~\cite{Bursztein2014The,Gao2016A}. Algorithm many text-based Captchas have proven insecure, the most recent studies~\cite{Thomas2013Trafficking,Bursztein2014Easy} suggest that Captcha is still a usefull security mechanism, and the security of text-based Captchas is currently a hot topic in the academic field.~\cite{Gao2017Research}

2. CAPTCHA is now a standard security technology for differentiating between computers and humans, and the most widely deployed schemes are text-based.

A main feature of such schemes is to use contour lines to form connected hollow characters with the aim of improving security and usability simultaneously, as it is hard for standard techniques to segment and recognize such connected characters, which are however easy to human eyes.

We show that with a simple but novel attack, we can successfully break a whole family of hollow CAPTCHAs, including those deployed by all the major companies. While our attack casts serious doubt on the viability of current designs, we offer lessons and guidelines for designing better hollow CAPTCHAs~\cite{Gao2013The}.

3. CAPTCHAs are automated tests to tell computers and humans apart. They are designed to be easily solvable by humans, but unsolvable by machines~\cite{Stark2015CAPTCHA}.

4. A CAPTCHA (Completely Automated Public Turing Test to Tell Computers and Human Apart) is a program that generates and grades tests that are human solvable, but intend to be beyond the capabilities of current computer programs~\cite{Von2004Telling}. This technology is now almost a standard security mechanism for defending against undesirable or malicious Internet bot programs, such as those spreading junk emails and those grabbing thousands of free email accounts instantly. It had found widespread application on numerous commercial web sites including Google, Yahoo, and Microsoft's MSN.

\textbf{Background} The most widely CAPTCHAs are the so-called text-based schemes, which rely on sophisticated distortion of text images aimed at rendering them unrecognisable to the state of the art of pattern recognition methods. The popularity of such schemes is due to the fact that they have many advantages~\cite{Chellapilla2005Building}, for example, being intuitive to users world-wide (the user task performed being just character recognition), having little localization issues (people in different countries all recognise Roman characters), and of good potential to provide strong security (e.g. the space a brute force attack has to search can be huge, if the scheme is properly designed).~\cite{Yan2008A} 
%\section{Related Work}
Our work lies at the intersection between adversarial machine learning and cracking text-based Captchas methods. This works brings together techniques developed in the domain of adversarial deep learning and character recognition to develop the attack on current text-based Captchas.

\noindent \textbf{Optical Character Recognition technology} Naor proposed the definition of Automated Turing Tests for automatically distinguishing humans from computer systems~\cite{Naor1996Verification}. This definition was firstly implemented by Lillibridge \emph{et al.}~\cite{Lillibridge2001Method} who developed the first practical scheme to prevent malicious bot programs from registering web accounts. However, such scheme was soon defeated by common used Optical Character Recognition (OCR) technique. Since then, a various of text-based and corresponding attacks have been proposed.

\noindent \textbf{Attacks on the individual Captchas} Mori~\emph{et al.}~\cite{Mori2003Recognizing} proposed an attack on Gimpy and EZ-Gimpy using complex object recognition algorithm. Their experiments shows that their approach can break 33\% Gimpy and 92\% EZ-Gimpy. Yan~\emph{et al.}~\cite{Yan2007Breaking} used a simple character segmentation method to break most Captchas provided by Captchaservices.org. They segmenting the character by counting the number of pixels of each individual character. Then they improved their character segmentation technique for attacking Microsoft-like Captcha schemes deployed by Yahoo!, Microsoft and Google~\cite{Yan2008A}. Gao~\emph{et al.}~\cite{Gao2013The} firstly proposed a generic method to break the hollow Captcha scheme. They first extracted the hollow character strokes based on color filling segmentation (CFS) method, and then searched for possible combinations of adjacent character strokes to form the individual characters. Recently they break Microsoft’s two-layer Captchas using similar approach with a high success rate~\cite{Gao2017Research}.
The above solutions first need to segment characters and then recognition them. But for the skewed and overlapped letters, their methods are hard to successfully segment as the characters are gathered together. To such scheme, Stark~\emph{et al.}~\cite{Stark2015CAPTCHA} presented an approach that can recognize the Captchas based on Active Deep Learning. Their approach can break Google's reCAPTCHA with a high success rate using a small number of labeled data.

\noindent \textbf{Generic attacks on text-based Captchas} Bursztein~\emph{et al.}~\cite{Bursztein2014The} reported a generic method to break a wide of text-based Captchas in a single step using machine learning algorithm. Their approach can recognize the Captcha by addressing the segmentation and recognition problems simultaneously. It scores all possible ways of character combinations to segment a captcha using reinforcement learning algorithm and search for the best possible one as the result.
More recently, Gao~\emph{et al.}~\cite{Gao2016A} present another generic approach that solved text-based Captchas. They first used Gabor filter to extract character components of the Captcha in four different directions, and then the extracted character components is ranked ordered by a graph search algorithm before recognizing the ranked components using Convolutional Neural Network.
The success of the two above approaches lies in the relative clean background. That is their approaches perfect poor when the Captcha has complex background as VISHOP scheme shown as Figure~\ref{fig:text-based captchas}. This is because their segmentation method cannot distinguish the individual character from such background.
George~\emph{et al.}~\cite{George2017A} proposed a novel network model named RCN that can break text-based Captchs using a samll number of training data.

\noindent \textbf{Study on the robustness of text-based Captchas} Bursztein~\cite{Bursztein2011Text} conducted a systematic study on existing text-based Captchas based on distorted characters. Among studying 15 Captcha deployed on popular websites, they found that 13 schemes excepting for Google and reCAPTCHA are vulnerable to automated attacks. They also explored whether it is easy to recognize the Captchas for human by conducting a large scale evaluation of Captchas from human's perspective~\cite{Bursztein2010How}. They collected 21 most widely used captcha schemes and asked the participants to manually recognize these Captchas. Their experiment results indicated that Captchas general were difficult for humans, especially for the audio Captcha schemes.
Krol~\emph{et al.}~\cite{Krol2016Better} carried out a user study on both reCAPTCHAs and "gamified" Captchas. They found that participants preferred to reCAPTCHAs than gamified Captchas because they were familiar with reCAPTCHAs.

\noindent \textbf{Attacks on other Captcha schemes} In addition to text-based Captchas, there are other Captchas such as image-based Captchas~\cite{Elson2007Asirra, Athanasopoulos2006Enhanced, Areyouhuman, Mohamed2017On, Gossweiler2009What}, audio-based Captchas~\cite{Schlaikjer2010A, Bigham2009Evaluating}.
However, researchers have proposed some attacks on image-based Captchas~\cite{Mohamed2014A, Gao2014An, Sivakorn2016I} and audio-based Captchas~\cite{Tam2008Breaking, Meutzner2014Reducing, Bursztein2009Decaptcha}, respectively. Further, such Captcha schemes are vulnerable to side-channel attacks~\cite{Hernandezcastro2009Side} as the limited number of example Captchas can be easily mined by the adversary.

\noindent \textbf{Adversarial machine learning} Recently, adversarial machine learning~\cite{Huang2011Adversarial} begins to be used in the field of information security. Researchers have found such technique can be used to evading classifiers of malicious software~\cite{Xu2016Automatically, Rosenberg2017Generic}, constructing specific malicious examples bypass intrusion detection system or spam e-mail filtering~\cite{Barreno2006Can} and cheat any classifiers trained by machine learning algorithms~\cite{Goodfellow2015Explaining, Miyato2015Distributional}.


%The Automated Turing Tests were first proposed by Naor~\cite{Naor1996Verification}, but they did not provide a formal definition. Lillibridge \emph{et al.}~\cite{Lillibridge2001Method} developed the first practical Automated Turing Test to prevent bots from automatically registering web pages. This system was effective for a while and then was defeated by common Optical Character Recognition (OCR) technology.
%
%Text-based Captchas based on English letters and Arabic numerals, is still the most widely deployed scheme. Many research communities focus on developing attack approaches for existing text-based Captchas, and then explore guidelines for better designs.
%
%A Captcha is considered broken if it can be automatically solved at a rate above 1\% at which Captchas are considered ineffective~\cite{Bursztein2011Text}. 

%\section*{Acknowledgements}
%    We would like to thank all participants who help for completing the experiments. Thank all volunteers for their time and insights as
%    well as the anonymous reviewer for their critical and constructive comments. This work was supported by NSFC (Grant No. 61672427) and
%    the UK Engineering and Physical Sciences Research Council (Grants No. EP/M01567X/1(SANDeRs) and EP/M015793/1(DIVIDEND)).

%\begin{spacing}{0.98}
\bibliographystyle{IEEEtranS}
\balance
\bibliography{refs}
%\end{spacing}

\end{document}
