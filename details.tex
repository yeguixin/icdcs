    \begin{algorithm}[!t]
        \centering
        \caption{Unifying the character font style}
        \label{alg:unify_font_style}
        \begin{algorithmic}[1]
            \REQUIRE~~\\
                $CI$: Captcha Image  \\
                $numChar$: The number of characters on Captchas\\
            \ENSURE~~\\
                $UC$: Captcha with the same font style \\
            \STATE $grayCI \leftarrow getBinaryImage(CI)$ \\
            \STATE $positions[] \leftarrow getHollowPositions(grayCI)$ \\
            \STATE $LEN \leftarrow getPositonsLen(positions[], numChar)$ \\
            \STATE $meanThick \leftarrow getMeanCharThick(CI, positions[])$ \\
            \FOR{$i=1:LEN$}
                \STATE $cFI \leftarrow FillRedColor(CI, positions[i])$ \\
                \STATE $cCI \leftarrow ChangeFillColor(cFI, positions[i])$ \\
                \STATE $charThick \leftarrow getCharThick(cCI, positions[i])$ \\
                \IF{$charThick>meanThick$}
                    \STATE $UC=cropChar(cCI, positions[i], meanThick)$ \\
                \ENDIF
            \ENDFOR
        \end{algorithmic}
    \end{algorithm}
\section{Implementation Details}
\section{Captcha Generator} \label{section: Captcha_generator}
A key success factor for deep learning system is large amount of data to train.
Intuitively, the simple yet direct approach in data collecting is to mine the Captchas from target websites and then paid to mark the labels to them artificially. Obviously, it is a time- or financial-consuming work with a high error rate during marking labels. Further, some type of Captchas are difficult to mine as the corresponding websites limit the accessing frequency.
In addition, there is never such a public Captcha generator which can produce enough different styles of Captchas that match the requirement of the training process.
Thus, the first concern of our work is to develop a flexible, well-calibrated training data generator.

To achieve it, we design an generation model to imitate Captchas production process, and automatically generate different styles of Captchas that are similar to those deployed in real-world websites.
Given a unique text-based Captcha scheme $x$, we first manually analyze the number of the characters and their corresponding security features, $S_{1:N}$, with $N$ parameters such as font style, size and color, rotating, distortion and waving \emph{et al.} described in Section~\ref{section: sccturity_features}.
Here we ranked the from No.1 to 10 shown as Table~\ref{table: feature_number}.
We specifics this analysis results as $\{ M, S_{1:N} \}$, where $M$ and $N$ respectively are the number of characters and its styles.
Then given the content of Captcha, our generation model is able to automatically produce Captchas image $y$, which accords with the above analyzed style.
We define our generation model as follows:
\begin{equation}\label{equation: generator_model}
  y \mid x \sim G_s(x),    x = \{M, N, L_{1:M}, S_{1:N} \}
\end{equation}

Where $G_s(x)$ is the generation model parameterized by unique style $s$, that can generate the Captcha image $y$ based on the unique Captcha scheme $x$. $M$ is the number of characters on the Captcha image and $L_{1:M}$ represents the content of the characters. For each character, our generation model individually generates corresponding style defined by $S_{1:N}$, because each character on some Captchas may have different style.

\textbf{Example:} We use the Captcha image depicted in Figure~\ref{fig:overview} as an example to describe our Captcha generator. This Captcha scheme consists of both English letters and Arabic numerals and the number of characters are fixed (4 characters). The content of the Captcha is $\{7, j, R, U\}$.
It uses 2 anti-segmentation features (Complex Background and Connection Lines) and 4 anti-recognition features (Character Set, Font size, Rotating and Distortion). So the collection of security features number is \{\circled{\small 0}, \circled{\small 1}, \circled{\small 3}, \circled{\small 4}, \circled{\small 6}, \circled{\small 7}\}.
Considering these parameters, the variance $x$ is $\{4, 6, \{7, j, R, U\}, \{\circled{\small 0}, \circled{\small 1}, \circled{\small 3}, \circled{\small 4}, \circled{\small 6}, \circled{\small 7}\}\}$.
For each character in the collection $\{4, 6, \{7, j, R, U\}$, our generator product the subgraph correspond to its security features. At last, the generator aggregates the subgraph to produce a Captcha image.



\begin{figure}
  \centering
  \includegraphics[width=0.5\textwidth]{fig/fill_color.pdf}
  \caption{The procedure of Captcha preprocessing.}
  \label{fig:fill_color}
\end{figure}

\begin{figure}
  \centering
  \includegraphics[width=0.45\textwidth]{fig/captcha_preprocessing/preprocessing.pdf}
  \caption{This figure shows how to unify the font style using the algorithm of morphological dilation.}
  \label{fig: preprocessing}
\end{figure}

\subsubsection{Captcha Preprocessing}
Given some current text-based Captcha schemes mainly consist of both solid characters and hollow characters (see Figure~\ref{fig:text-based captchas} (j) and (k)), the first step of our attack is to unify the different font styles to a fixed style. Here we aims to fill the hollow character with solid core due to the following two reasons:
(1) the hollow character can be easily transformed to the solid one but not the opposite.
(2) the solid character can be extracted more stable features than hollow character at the following step according to our preliminary experiments.

To do so, we first convert the colorful image to black-and-white using the standard threshold selection method proposed by Otsu~\cite{Ostu1979A}\footnote{Note that we only use the binarized image to locate the position of the hollow part of the character other than furthering processing.}.
Then the Color Filling Segmentation (CFS)~\cite{Yan2008A} is used to fill the hollow character with the red color (see Figure~\ref{fig:fill_color} (b)). Next, the color-filled image should be convert to color-coincident image by changing the red filled area to the original character color (see Figure~\ref{fig:fill_color} (c)). At last, the thick filled characters on the color-coincident image need to be cropped as the original character (see Figure~\ref{fig:fill_color} (d)).

However, how to translate the color-coincident image to the target image (shown in Figure~\ref{fig:fill_color}) is a challenge. Fugure~\ref{fig: preprocessing} presents our approach to address this challenge. We first segment the characters using CFS method~\cite{Yan2008A}.  Then we extract the color-filled characters and using morphological algorithm to dilate these characters. To effectively dilate both the edge and inflexion areas, we use the ellipse and cross kernels. We set the size of the kernel to (10, 10), an empirical setting that makes dilated characters clearly visible. After that, we combine these segmented characters to a Captcha image.

Our method for unifying the character font style is described in Algorithm~\ref{alg:unify_font_style}.
The input to the algorithm is the original Captcha image with different font styles and the number of characters on this Captcha, and the output of the algorithm is the Captcha with solid characters. To locate the hollow characters, we first convert the colorful original image to binarized image (line 1). According to the binarized image, we can get the positions of the each hollow characters on the colorful Captcha image (line 2) because the size of colorful image is the same as the binarized image. For each hollow character, we use CFS method described above to fill it with red color (line 6) and then replace the red filled color with the character color to get the color-coincident image (line 7). Finally, we crop the filled character on the color-coincident image to unify the font style of characters (line 10).

\begin{figure}
  \centering
  \includegraphics[width=0.3\textwidth]{fig/loss_anlysis/captcha_generated.pdf} \\
  \caption{Captchas generated when using L1 and L2 loss function.}
  \label{fig: L1_L2}
\end{figure}

\begin{figure}
  \centering
  \subfigure{
      \begin{minipage}[t]{0.22\textwidth}
      \includegraphics[width=\textwidth]{fig/loss_anlysis/loss_L1.pdf}\\
      \center L1 loss curve
      \end{minipage}
  }
  \subfigure{
      \begin{minipage}[t]{0.22\textwidth}
      \includegraphics[width=\textwidth]{fig/loss_anlysis/loss_L2.pdf}\\
      \center L2 loss curve
      \end{minipage}
  }
  \caption{This figure shows the L1 and L2 loss curve when train using Captcha scheme in Figure~\ref{fig: L1_L2}.}
  \label{fig: loss_anlysis}
\end{figure}

\subsection{Generate Regular Captcha}
After unifying the character font style, we need to generate regular Captchas for further recognizing. We achieve this by employing an image-to-image translation algorithm called~\emph{Pix2Pix}~\cite{Pix2PixCode}. This algorithm automatically translate the image from the original style to the target style. In our case, the images to be translated are the Captchas with complex noisy background, overlapping and distorted characters (original style). These are supplied to the algorithm by a Captcha generator developed using a simple script (Section~\ref{section: Captcha_generator}). The algorithm tries to generate regular Captcha with appropriate character spacing and clean background (target style).
Note that if the Captcha to be translated has no background as Google scheme (Figure~\ref{fig:text-based captchas} (l)), our approach just enlarge the space of adjacent character and Regularization operations. Likewise, if the Captcha only contains deformed but no overlapping characters (Figure~\ref{fig:text-based captchas} (j)), our approach only need to regular the characters.

\noindent \textbf{Hierarchical Methods.} In order to generate regular Captcha, we propose a hierarchical method that employ a variant approach of \emph{Pix2Pix} to complete the progressive tasks.
The hierarchical methods are comprised of three sequenced models, and they can respectively achieve three different tasks: removing complex background, expanding space of adjacent overlapping characters and making the distorted character to be a regular one. These three sequence models share the same methodology of~\emph{Pix2pix}.
The main differences of these sequences models are the input and output.
The output of the previous model is the input of the following model.
Take the Captcha in Figure~\ref{fig:overview} for an example, the first model is applied to eliminate the complex background and Connecting lines and output the Captcha with clean background. Then the space of the characters on the output Captcha is expanded by the second model, and produces the Captcha only with distorted characters. Finally, the distorted characters are translated to the regular ones using the last model.
%The inputs of the first step are the images with complex background, overlapping and distorted characters.
%These inputting images are eliminated the complex background by the variant approach which outputs the Captcha images with clean background, overlapping and distorted characters.
%According to the outputs, the second step is to expand the space between the adjacent overlapping characters using the same variant method, and produces the Captcha images only with distorted characters.
%Finally, the distorted characters are translated to the regular ones using the variant method.

\noindent \textbf{Key Algorithm.} The key part of the hierarchical methods is a variant of \emph{Pix2Pix}. It consists of an image generator and an discriminator. They are competing with each other until reach Nash equilibrium during training processing (Section~\ref{section: GANs}).
We take the generation model of removing complex background for detailed introducing our key algorithm.
The goal is to train a generative model that can translate the Captcha image with complex background to the images with clean background.
In order to train this generation model, the input data are the image pairs including the Captcha images with background and the images with clean background.
During training, the image generator produces the fake image that similar to the image with white background by randomly adding the noisy points to it. The fake image and the image with complex background compose an image pair.
For the composed image pair, the discriminator learns to classify between the real and fake pairs. This competing process will be terminated until the discriminator cannot correctly classify the image pairs.
Unlike the \emph{Pix2Pix}, we use the L2 loss function to figure out the loss of the generator because L2 loss function can capturing the overall structure of the Captcha image, which contribute to removing the background and other two tasks (expand the space of adjacent characters and eliminate distorted characters).
\begin{equation}\label{equation: L2_loss}
    \mathcal{L}_{L2}(Gen) = \mathbb{E}_{x,y \epsilon C_{O}, z \epsilon N_{O}} \|y - Gen(x, z)\|_{2}
\end{equation}

Where $x$ is the Captcha image with white background, $y$ is the image with complex background and $z$ is the fake image with the noisy points.

\noindent \textbf{Analysis of Loss Function.} To demonstrate the effectiveness of L2 over L1 loss function, we conduct many perliminary experiments. Take Figure~\ref{fig: L1_L2} (a) for instance, it is expected to be translated to regular Captcha as Figure~\ref{fig: L1_L2} (b). When using L1 loss function, only one character can be successful translated to the target one shown as Figure~\ref{fig: L1_L2} (c). In contrary, all characters can be correctly translated to be the regular ones (Figure~\ref{fig: L1_L2} (d)) when using L2 loss function. This is because the generation model can achieve global optimum when using L2 loss function. Figure~\ref{fig: loss_anlysis} shows that the loss value of L1 has been fluctuating while the value of L2 has been tending to convergence.

%The hierarchical methods are comprised of three sequenced models, and they can respectively achieve the tasks of removing complex background, expanding space between adjacent overlapping characters and translating the distorted Captcha to a regular one. The sequenced models share the same translation model.
%The key part of the translation model is a variant of \emph{Pix2Pix}, which consists of a image generator and discriminator. They are competing with each other until reach Nash equilibrium when training processing (Section~\ref{section: GANs}).
%
%In our case of removing background, the inputting data are the image pairs including the Captcha images with complex background and the images with white background.
%The goal is to train a generative model that can translate the Captcha image with complex background to the images with white background.

\begin{figure}[!t]
  \centering
  \includegraphics[width=0.45\textwidth]{fig/cnn_model.pdf}
  \caption{CNN recognition engine. The input of the recognition recognize is the regular Captcha image, and it output the text of the Captcha image.}
  \label{fig: cnn_model}
\end{figure}

\subsection{Identify Captcha}
In this step, we use a rudimentary CNN framework, \emph{LeNet-5}~\cite{Lecun1998Gradient}, as our recognition engine, to identify the text of the translated Captcha image.
The initial \emph{Lenet-5} was compromised of three convolutional layers, two pooling layers and followed by two fully connected layers.
The convolutional layer extracts the features using a number of filters that are trained during training process. The pooling layer aggregates the features extracted from the convolutional layer for extracting more representative features meanwhile reducing the amount of calculation. The fully connected layer classify the extracted features into target categories. The appropriate number of network layers determines the quality of the extracted features as proper number of layers will extract more representative features.

Unlike the \emph{LeNet-5}, the goal of our recognition engine is to recognize the text of Captchas on the whole, which is more difficult than recognition a single character done by \emph{LeNet-5}. This is because recognizing more characters need to extract more complex and abstract features.
To do so, we redesign the \emph{LeNet-5} and adding another two convolution layers and another three pooling layers.
Figure~\ref{fig: cnn_model} depicts the framework of our recognition engine. Generally, it consists of five convolution layers, five pooling layers and followed by two fully connected layers. Each convolution layer is followed by a pooling layer.

To extract more representative features, each convolutional layer uses a convolution filter of $3 \times 3$ and each pooling layer employs the max-pooling value. Other parameters are the same as the \emph{LeNet-5}.











