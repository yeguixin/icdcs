\section{Related Work}
Our work lies at the intersection between adversarial machine learning and cracking text-based Captchas methods. This works brings together techniques developed in the domain of adversarial deep learning and character recognition to develop the attack on current text-based Captchas.

\noindent \textbf{Optical Character Recognition technology} Naor proposed the definition of Automated Turing Tests for automatically distinguishing humans from computer systems~\cite{Naor1996Verification}. This definition was firstly implemented by Lillibridge \emph{et al.}~\cite{Lillibridge2001Method} who developed the first practical scheme to prevent malicious bot programs from registering web accounts. However, such scheme was soon defeated by common used Optical Character Recognition (OCR) technique. Since then, a various of text-based and corresponding attacks have been proposed.

\noindent \textbf{Attacks on the individual Captchas} Mori~\emph{et al.}~\cite{Mori2003Recognizing} proposed an attack on Gimpy and EZ-Gimpy using complex object recognition algorithm. Their experiments shows that their approach can break 33\% Gimpy and 92\% EZ-Gimpy. Yan~\emph{et al.}~\cite{Yan2007Breaking} used a simple character segmentation method to break most Captchas provided by Captchaservices.org. They segmenting the character by counting the number of pixels of each individual character. Then they improved their character segmentation technique for attacking Microsoft-like Captcha schemes deployed by Yahoo!, Microsoft and Google~\cite{Yan2008A}. Gao~\emph{et al.}~\cite{Gao2013The} firstly proposed a generic method to break the hollow Captcha scheme. They first extracted the hollow character strokes based on color filling segmentation (CFS) method, and then searched for possible combinations of adjacent character strokes to form the individual characters. Recently they break Microsoft’s two-layer Captchas using similar approach with a high success rate~\cite{Gao2017Research}.
The above solutions first need to segment characters and then recognition them. But for the skewed and overlapped letters, their methods are hard to successfully segment as the characters are gathered together. To such scheme, Stark~\emph{et al.}~\cite{Stark2015CAPTCHA} presented an approach that can recognize the Captchas based on Active Deep Learning. Their approach can break Google's reCAPTCHA with a high success rate using a small number of labeled data.

\noindent \textbf{Generic attacks on text-based Captchas} Bursztein~\emph{et al.}~\cite{Bursztein2014The} reported a generic method to break a wide of text-based Captchas in a single step using machine learning algorithm. Their approach can recognize the Captcha by addressing the segmentation and recognition problems simultaneously. It scores all possible ways of character combinations to segment a captcha using reinforcement learning algorithm and search for the best possible one as the result.
More recently, Gao~\emph{et al.}~\cite{Gao2016A} present another generic approach that solved text-based Captchas. They first used Gabor filter to extract character components of the Captcha in four different directions, and then the extracted character components is ranked ordered by a graph search algorithm before recognizing the ranked components using Convolutional Neural Network.
The success of the two above approaches lies in the relative clean background. That is their approaches perfect poor when the Captcha has complex background as VISHOP scheme shown as Figure~\ref{fig:text-based captchas}. This is because their segmentation method cannot distinguish the individual character from such background.
George~\emph{et al.}~\cite{George2017A} proposed a novel network model named RCN that can break text-based Captchs using a samll number of training data.

\noindent \textbf{Study on the robustness of text-based Captchas} Bursztein~\cite{Bursztein2011Text} conducted a systematic study on existing text-based Captchas based on distorted characters. Among studying 15 Captcha deployed on popular websites, they found that 13 schemes excepting for Google and reCAPTCHA are vulnerable to automated attacks. They also explored whether it is easy to recognize the Captchas for human by conducting a large scale evaluation of Captchas from human's perspective~\cite{Bursztein2010How}. They collected 21 most widely used captcha schemes and asked the participants to manually recognize these Captchas. Their experiment results indicated that Captchas general were difficult for humans, especially for the audio Captcha schemes.
Krol~\emph{et al.}~\cite{Krol2016Better} carried out a user study on both reCAPTCHAs and "gamified" Captchas. They found that participants preferred to reCAPTCHAs than gamified Captchas because they were familiar with reCAPTCHAs.

\noindent \textbf{Attacks on other Captcha schemes} In addition to text-based Captchas, there are other Captchas such as image-based Captchas~\cite{Elson2007Asirra, Athanasopoulos2006Enhanced, Areyouhuman, Mohamed2017On, Gossweiler2009What}, audio-based Captchas~\cite{Schlaikjer2010A, Bigham2009Evaluating}.
However, researchers have proposed some attacks on image-based Captchas~\cite{Mohamed2014A, Gao2014An, Sivakorn2016I} and audio-based Captchas~\cite{Tam2008Breaking, Meutzner2014Reducing, Bursztein2009Decaptcha}, respectively. Further, such Captcha schemes are vulnerable to side-channel attacks~\cite{Hernandezcastro2009Side} as the limited number of example Captchas can be easily mined by the adversary.

\noindent \textbf{Adversarial machine learning} Recently, adversarial machine learning~\cite{Huang2011Adversarial} begins to be used in the field of information security. Researchers have found such technique can be used to evading classifiers of malicious software~\cite{Xu2016Automatically, Rosenberg2017Generic}, constructing specific malicious examples bypass intrusion detection system or spam e-mail filtering~\cite{Barreno2006Can} and cheat any classifiers trained by machine learning algorithms~\cite{Goodfellow2015Explaining, Miyato2015Distributional}.


%The Automated Turing Tests were first proposed by Naor~\cite{Naor1996Verification}, but they did not provide a formal definition. Lillibridge \emph{et al.}~\cite{Lillibridge2001Method} developed the first practical Automated Turing Test to prevent bots from automatically registering web pages. This system was effective for a while and then was defeated by common Optical Character Recognition (OCR) technology.
%
%Text-based Captchas based on English letters and Arabic numerals, is still the most widely deployed scheme. Many research communities focus on developing attack approaches for existing text-based Captchas, and then explore guidelines for better designs.
%
%A Captcha is considered broken if it can be automatically solved at a rate above 1\% at which Captchas are considered ineffective~\cite{Bursztein2011Text}. 